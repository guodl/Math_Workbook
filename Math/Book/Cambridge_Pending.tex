


\chapterimage{chapter_head_2.pdf}
\chapter{Skill List}


\section{Algebra}

111Simply $x^2-5+3x$ = \begin{boxAns}[width=10cm]$x^2=5$\end{boxAns}  \\


boxAns: 222Simply $x^2-5+3x$ = \begin{boxAns}$x^2=5$\end{boxAns}  \\


boxAnswer: Simply $x^2-5+3x$ = \boxAnswer[width=10cm]{$x^2=5$}  \\

Simply $x^2-5+3x$ = \boxAnswer{$x^2=5$}
Simply $x^2-5+3x$ = \boxAnswer[width=10cm]{$x^2=5$} \\
Simply $x^2-5+3x$ = \boxAnswer{$x^2=5$}
Simply $x^2-5+3x$ = \boxAnswer{$x^2=5$}







%-----------------------------------------------------------------------------------------
%   https://tex.stackexchange.com/questions/287892/setcounter-in-tasks-package
%
%
%
%
%
%-----------------------------------------------------------------------------------------


\setcounter{exercise}{722}

\setcounter{task}{299}


\begin{tasks}[resume](2)
    \task 
          \begin{exercise}[resume] First task  
  \begin{envSolution}
 
      sdsdsds


          
   $ \frac{ab}{ \frac{1} { dsdds d ds s  }}=c  $\\
   $ \frac{ab}{dsdds d ds s}=c  $\\
      
  
  \begin{shortsolution}
  	$ (x+5)^3+c $
  \end{shortsolution}
  \end{envSolution}%
 \end{exercise}
 
    \task Second task
\begin{exercise}
	
    
  \begin{envSolution}
 
      sdsdsds


          
   $ \frac{ab}{ \frac{1} { dsdds d ds s  }}=c  $\\
   $ \frac{ab}{dsdds d ds s}=c  $\\
      
  \begin{shortsolution}
  	$ \sqrt{x+5}+c $
  \end{shortsolution}
  
  \end{envSolution}%

\end{exercise}

 
    \task Third task spanning the full width available
    \task Fourth task
 \end{tasks}











\setcounter{exercise}{322}
\begin{tasks}[start=320](2)
    \task First task
          \begin{exercise}  \SetExerciseProperties{counter=\thetask}
  \begin{envSolution}
 
      sdsdsds


          
   $ \frac{ab}{ \frac{1} { dsdds d ds s  }}=c  $\\
   $ \frac{ab}{dsdds d ds s}=c  $\\
      
  
  \begin{shortsolution}
  	$ (x+5)^3+c $
  \end{shortsolution}
  \end{envSolution}%
 \end{exercise}
 
    \task Second task
\begin{exercise}
	
    
  \begin{envSolution}
 
      sdsdsds


          
   $ \frac{ab}{ \frac{1} { dsdds d ds s  }}=c  $\\
   $ \frac{ab}{dsdds d ds s}=c  $\\
      
  \begin{shortsolution}
  	$ \sqrt{x+5}+c $
  \end{shortsolution}
  
  \end{envSolution}%

\end{exercise}

 
    \task Third task spanning the full width available
    \task Fourth task
 \end{tasks}


 \begin{tasks}(2)
    \task First task
    \task Second task
    \task Third task spanning the full width available
    \task Fourth task
 \end{tasks}



\clearpage 


\begin{enumerate}[resume]
	\setcounter{exercise}{099} % Should set exercise number 189 after enumberate
	




\begin{enumerate}[resume]
	\setcounter{exercise}{189} % Should set exercise number 189 after enumberate
	
	
	\begin{exercise} Exercise 356 $2sinx+3cosx$. 
		\begin{envSolution}
			
			sdsdsds
			
			
			
			$ \frac{ab}{ \frac{1} { dsdds d ds s  }}=c  $\\
			$ \frac{ab}{dsdds d ds s}=c  $\\
			
			
			\begin{shortsolution}
				No. \theexercise Short solution  $\int x^2 + 3x dx $
			\end{shortsolution}
			
		\end{envSolution}%
		
	\end{exercise}















	
	\begin{exercise} Exercise 356 $2sinx+3cosx$. 
		\begin{flalign*}
			\cmdSolution{123 =2cosx-3sinx  }   
			\cmdSolution{  }   
			\cmdSolution{ }   
			\cmdAnswer{ }   
		\end{flalign*}
	\end{exercise}







\begin{exercise}	
	 Exercise 357 $2sinx+3cosx$. 
	\begin{flalign*}
		\cmdSolution{=2cosx-3sinx  }   
		\cmdSolution{  }   
		\cmdSolution{ }   
		\cmdAnswer{ }   
	\end{flalign*}
	
	
			\begin{shortsolution}
		\ Short solution  $\int x^2 + 3x dx $
	\end{shortsolution}
	
\end{exercise}

\end{enumerate}


















 \section{Algebra - enumerate resume}


    \setcounter{enumi}{978}
	
\begin{enumerate}[resume]

	\item $ 979  2sinx+3cosx$.
	\begin{flalign*}
		\cmdSolution{$121 =2cosx-3sinx $ }   
		\cmdSolution{ a }   
		\cmdSolution{ b}   
		\cmdAnswer{ }   
	\end{flalign*}



	\item $ 979  2sinx+3cosx$.
	\begin{flalign*}
		\cmdSolution{$121 =2cosx-3sinx $ }   
		\cmdSolution{ a }   
		\cmdSolution{ b}   
		\cmdAnswer{ }   
	\end{flalign*}

	\item 980.12 $2sinx+3cosx$.
    \begin{flalign*}
	    \cmdSolution{$123 =2cosx-3sinx $ }   
   	    \cmdSolution{ c }   
	    \cmdSolution{d }   
	   \cmdAnswer{ }   
    \end{flalign*}


	\item 912.32 $2sinx+3cosx$.\\ 
    \begin{flalign*}
	    \cmdSolution{=2cosx-3sinx  }   
   	    \cmdSolution{  }   
	    \cmdSolution{ }   
	   \cmdAnswer{ }   
    \end{flalign*}
		
\end{enumerate}


% \cmdRef{3}{../../../Education/ExamPaper/Cambridge_Ages_16_19_Cambridge_Advanced/Mathematics_9709/9709_m19_qp_32.pdf}
% \cmdRef{4}{../../../Education/ExamPaper/Cambridge_Ages_16_19_Cambridge_Advanced/Mathematics_9709/9709_m19_qp_32.pdf}
% \cmdRef{5}{../../../Education/ExamPaper/Cambridge_Ages_16_19_Cambridge_Advanced/Mathematics_9709/9709_m19_qp_32.pdf}
% \cmdRef{6}{../../../Education/ExamPaper/Cambridge_Ages_16_19_Cambridge_Advanced/Mathematics_9709/9709_m19_qp_32.pdf}
% \cmdRef{7}{../../../Education/ExamPaper/Cambridge_Ages_16_19_Cambridge_Advanced/Mathematics_9709/9709_m19_qp_32.pdf}
% \cmdRef{8}{../../../Education/ExamPaper/Cambridge_Ages_16_19_Cambridge_Advanced/Mathematics_9709/9709_m19_qp_32.pdf}

\section{Algebra - task - Exercise}


\setcounter{cntTask}{342} 

% \begin{tasks}[resume](2)



	% \task  In  task falign Task \thecntTask $\frac{x^2}{x+3} + 2sinx+3cosx$.\\
	  % \begin{flalign*}  
	      % \textcolor{white}{.} 111  & =ssss  & f123 \textcolor{white}{.} 
	     % \textcolor{white}{.} 111  & =ssss  & \textcolor{white}{.} \\
   	       % x^2        & =(a+1)       \; & \frac{1}{x+1}\\
           % x^3        & a           \; & \frac{1}{x+1} \\
		   % 123        & 2cosx-3sinx \;  & a \\
		   % 1          &=2cosx-3sinx  \;          & 1\\
		  % 1		   & =2cosx-3sinx \; &23
        % \end{flalign*}



	% \task  In  task falign Task \thecntTask $\frac{x^2}{x+3} + 2sinx+3cosx$.\\
        % \begin{flalign*}  
   	       % \cmdSolution{ 123 &=2cosx-3sinx }
		   % \cmdSolution{&=2cosx-3sinx }
		   % \cmdSolution{&=2cosx-3sinx }
		   % \cmdSolution{&=2cosx-3sinx }
		   % \cmdSolution{& =2cosx-3sinx }	   		   
        % \end{flalign*}


% \end{tasks}

	% \task 23 Task \thecntTask $\frac{x^2}{x+3} + 2sinx+3cosx$. 
        % \begin{flalign*}  
   	       % \cmdSolution{$=2cosx-3sinx$ }
		   % \cmdSolution{$=2cosx-3sinx$ }
		   % \cmdSolution{$=2cosx-3sinx$ }
		   % \cmdSolution{$=2cosx-3sinx$ }
		   % \cmdSolution{$=2cosx-3sinx$ }
		   		   
        % \end{flalign*}



	% \task Task \thecntTask $\frac{x^2}{x+3} + 2sinx+3cosx$. 
        % \begin{flalign*}  
   	       % \cmdSolution{$=2cosx-3sinx$ }
		   % \cmdSolution{$=2cosx-3sinx$ }
		   % \cmdSolution{$=2cosx-3sinx$ }
		   % \cmdSolution{$=2cosx-3sinx$ }
		   % \cmdSolution{$=2cosx-3sinx$ }
		   		   
        % \end{flalign*}


	% \task Task \thecntTask $\frac{x^2}{x+3} + 2sinx+3cosx$. 
        % \begin{flalign*}  
   	       % \cmdSolution{$=2cosx-3sinx$ }
		   % \cmdSolution{$=2cosx-3sinx$ }
		   % \cmdSolution{$=2cosx-3sinx$ }
		   % \cmdSolution{$=2cosx-3sinx$ }
		   % \cmdSolution{$=2cosx-3sinx$ }
		   		   
        % \end{flalign*}



	% \task Task \thecntTask $\frac{x^2}{x+3} + 2sinx+3cosx$. 
        % \begin{flalign*}  
   	       % \cmdSolution{$=2cosx-3sinx$ }
		   % \cmdSolution{$=2cosx-3sinx$ }
		   % \cmdSolution{$=2cosx-3sinx$ }
		   % \cmdSolution{$=2cosx-3sinx$ }
		   % \cmdSolution{$=2cosx-3sinx$ }
		   		   
        % \end{flalign*}


	% \task Task \thecntTask $\frac{x^2}{x+3} + 2sinx+3cosx$. 
        % \begin{flalign*}  
   	       % \cmdSolution{$=2cosx-3sinx$ }
		   % \cmdSolution{$=2cosx-3sinx$ }
		   % \cmdSolution{$=2cosx-3sinx$ }
		   % \cmdSolution{$=2cosx-3sinx$ }
		   % \cmdSolution{$=2cosx-3sinx$ }
		   		   
        % \end{flalign*}



	% \task Task \thecntTask $\frac{x^2}{x+3} + 2sinx+3cosx$. 
        % \begin{flalign*}  
   	       % \cmdSolution{$=2cosx-3sinx$ }
		   % \cmdSolution{$=2cosx-3sinx$ }
		   % \cmdSolution{$=2cosx-3sinx$ }
		   % \cmdSolution{$=2cosx-3sinx$ }
		   % \cmdSolution{$=2cosx-3sinx$ }
		   		   
        % \end{flalign*}


	% \task Task \thecntTask $\frac{x^2}{x+3} + 2sinx+3cosx$. 
        % \begin{flalign*}  
   	       % \cmdSolution{$=2cosx-3sinx$ }
		   % \cmdSolution{$=2cosx-3sinx$ }
		   % \cmdSolution{$=2cosx-3sinx$ }
		   % \cmdSolution{$=2cosx-3sinx$ }
		   % \cmdSolution{$=2cosx-3sinx$ }
		   		   
        % \end{flalign*}


% \end{tasks}

% \noindent task - Exercise Differentiate with respect to $x$.


% \setcounter{cntTask}{42} 

% \begin{tasks}[resume](2)


% \begin{exercise} Exercise 1234 $2sinx+3cosx$. 
	% \begin{flalign*}
		% \cmdSolution{=2cosx-3sinx  }   
		% \cmdSolution{  }   
		% \cmdSolution{ }   
		% \cmdAnswer{ }   
	% \end{flalign*}
	
	% $2sinx+3cosx$. 
	% \begin{flalign*}
		% \cmdSolution{=2cosx-3sinx  }   
		% \cmdSolution{  }   
		% \cmdSolution{ }   
		% \cmdAnswer{ }   
	% \end{flalign*}
	
% \end{exercise}




% \begin{exercise} Task 1235 $2sinx+3cosx$. 
	% \begin{flalign*}
		% \cmdSolution{=2cosx-3sinx  }   
		% \cmdSolution{  }   
		% \cmdSolution{New Solution 3\sinx \cosx }   
		% \cmdAnswer{ }   
	% \end{flalign*}
	


	% \begin{flalign*}
		% \cmdSolution{=2cosx-3sinx  }   
		% \cmdSolution{  }   
		% \cmdSolution{New Solution 456 3\sinx \cosx }   
		% \cmdAnswer{ }   
	% \end{flalign*}
	

	
	% \begin{shortsolution}
		% $ in Same   /int x^2 + 3x dx $
	% \end{shortsolution}
	
% \end{exercise}




	% \task Task $\frac{x^2}{x+3} + 2sinx+3cosx$. 
        % \begin{flalign*}  
   	       % \begin{envSol}$=2cosx-3sinx$ \end{envSol}
		   % \begin{envSol}$=2cosx-3sinx$ \end{envSol}
		   % \begin{envSol}$=2cosx-3sinx$ \end{envSol}
		   % \begin{envSol}$=2cosx-3sinx$ \end{envSol}
		   % \begin{envSol}$=2cosx-3sinx$ \end{envSol}
		   
        % \end{flalign*}

	% \task $2sinx+3cosx$. 
        % \begin{flalign*}  
   	       % \begin{envSol}$=2cosx-3sinx$ \end{envSol}
		   % \begin{envSol}$=2cosx-3sinx$ \end{envSol}
		   % \begin{envSol}$=2cosx-3sinx$ \end{envSol}
		   % \begin{envSol}$=2cosx-3sinx$ \end{envSol}
		   % \begin{envSol}$=2cosx-3sinx$ \end{envSol}
		   
        % \end{flalign*}

	% \task $2sinx+3cosx$. 
	% \begin{flalign*}
		% \cmdSolution{=2cosx-3sinx  }   
		% \cmdSolution{  }   
		% \cmdSolution{ }   
		% \cmdAnswer{ }   
	% \end{flalign*}
	
	% \task $2sinx+3cosx$. 
	% \begin{flalign*}
		% x^2 & x^3\\
		% \frac{1}{x+1}
		% \cmdSolution{=2cosx-3sinx  }   
		% \cmdSolution{  }   
		% \cmdSolution{ }   
		% \cmdAnswer{ }   
	% \end{flalign*}
	
	% \task $2sinx+3cosx$. 
    % \begin{flalign*}
	    % \cmdSolution{=2cosx-3sinx  }   
	    % \cmdSolution{  }   
	    % \cmdSolution{New Solution 3\sinx \cosx }   
	    % \cmdAnswer{ }   
    % \end{flalign*}

% \end{tasks}


% \section{Algebra - enumerate resume}


    % \setcounter{enumi}{978}
	
% \begin{enumerate}[resume]

	% \item $ 979  2sinx+3cosx$.
	% \begin{flalign*}
		% \cmdSolution{$121 =2cosx-3sinx $ }   
		% \cmdSolution{ a }   
		% \cmdSolution{ b}   
		% \cmdAnswer{ }   
	% \end{flalign*}



	% \item $ 979  2sinx+3cosx$.
	% \begin{flalign*}
		% \cmdSolution{$121 =2cosx-3sinx $ }   
		% \cmdSolution{ a }   
		% \cmdSolution{ b}   
		% \cmdAnswer{ }   
	% \end{flalign*}

	% \item 980.12 $2sinx+3cosx$.
    % \begin{flalign*}
	    % \cmdSolution{$123 =2cosx-3sinx $ }   
   	    % \cmdSolution{ c }   
	    % \cmdSolution{d }   
	   % \cmdAnswer{ }   
    % \end{flalign*}


	% \item 912.32 $2sinx+3cosx$.\\ 
    % \begin{flalign*}
	    % \cmdSolution{=2cosx-3sinx  }   
   	    % \cmdSolution{  }   
	    % \cmdSolution{ }   
	   % \cmdAnswer{ }   
    % \end{flalign*}
		
% \end{enumerate}

% \noindent
% Differentiate with respect to $x$.

% \begin{enumerate}[resume]
	% \item $2sinx+3cosx$. 
	% \begin{flalign*}
		% \cmdSolution{=2cosx-3sinx  }   
		% \cmdSolution{  }   
		% \cmdSolution{ }   
		% \cmdAnswer{ }   
	% \end{flalign*}
	
% \end{enumerate}


% \section{Algebra - exercise - Update 2188}


% \NumTabs{2}

% \begin{enumerate}
	

% \setcounter{exercise}{200}

% \begin{exercise} Exercise $2sinx+3cosx$. 
	% \begin{flalign*}
		% \cmdSolution{2345 yue &=2cosx-3sinx  }   
		% \cmdSolution{  }   
		% \cmdSolution{New Solution 3 &= \sinx \cosx }   
		% \cmdAnswer{ }   
	% \end{flalign*}

% \end{exercise}


% \begin{exercise} Exercise 1.23 $2sinx+3cosx$. 
	
	% $2sinx+3cosx$. 
	% \begin{flalign*}
		% \cmdSolution{=2cosx-3sinx  }   
		% \cmdSolution{  }   
		% \cmdSolution{ }   
		% \cmdAnswer{ }   
	% \end{flalign*}
	
% \end{exercise}




% \begin{exercise} $2sinx+3cosx$. 
	% \begin{flalign*}
		% \cmdSolution{=2cosx-3sinx  }   
		% \cmdSolution{  }   
		% \cmdSolution{ }   
		% \cmdAnswer{ }   
	% \end{flalign*}
	


	% \begin{flalign*}
		% \cmdSolution{=2cosx-3sinx  }   
		% \cmdSolution{  }   
		% \cmdSolution{ }   
		% \cmdAnswer{ }   
	% \end{flalign*}
	

	
	% \begin{shortsolution}
		% $ in Same   /int x^2 + 3x dx $
	% \end{shortsolution}
	
% \end{exercise}

% \begin{exercise} $2sinx+3cosx$. 
	% \begin{flalign*}
		% \cmdSolution{=2cosx-3sinx  }   
		% \cmdSolution{  }   
		% \cmdSolution{ }   
		% \cmdAnswer{ }   
	% \end{flalign*}
	
	 % $2sinx+3cosx$. 
	% \begin{flalign*}
		% \cmdSolution{=2cosx-3sinx  }   
		% \cmdSolution{  }   
		% \cmdSolution{New Solution 3\sinx \cosx }   
		% \cmdAnswer{ }   
	% \end{flalign*}
	
	

	
	% \begin{shortsolution}
	% $ 1234567890  /int x^2 + 3x dx $
% \end{shortsolution}
	
% \end{exercise}


% \begin{exercise} $2sinx+3cosx$. 
	% \begin{flalign*}
		% \cmdSolution{=2cosx-3sinx  }   
		% \cmdSolution{  }   
		% \cmdSolution{ }   
		% \cmdAnswer{ }   
	% \end{flalign*}

	
	% item $2sinx+3cosx$. 
	% \begin{flalign*}
		% \cmdSolution{=2cosx-3sinx  }   
		% \cmdSolution{  }   
		% \cmdSolution{ }   
		% \cmdAnswer{ }   
	% \end{flalign*}

	% \begin{shortsolution}
	    % $3cosx +4 sinx$
    % \end{shortsolution}

	
	% \begin{solution}
		% $3cosx +4 sinx$
	% \end{solution}


% \end{exercise}


% \end{enumberate}


 \clearpage

 \section{Short Solutions}


\printshortsolutions

 \clearpage

 \section{Solutions}


 \printsolutions


% \cmdRef{3}{../../../Education/ExamPaper/Cambridge_Ages_16_19_Cambridge_Advanced/Mathematics_9709/9709_m19_qp_32.pdf}
% \cmdRef{4}{../../../Education/ExamPaper/Cambridge_Ages_16_19_Cambridge_Advanced/Mathematics_9709/9709_m19_qp_32.pdf}
% \cmdRef{5}{../../../Education/ExamPaper/Cambridge_Ages_16_19_Cambridge_Advanced/Mathematics_9709/9709_m19_qp_32.pdf}
% \cmdRef{6}{../../../Education/ExamPaper/Cambridge_Ages_16_19_Cambridge_Advanced/Mathematics_9709/9709_m19_qp_32.pdf}
% \cmdRef{7}{../../../Education/ExamPaper/Cambridge_Ages_16_19_Cambridge_Advanced/Mathematics_9709/9709_m19_qp_32.pdf}
% \cmdRef{8}{../../../Education/ExamPaper/Cambridge_Ages_16_19_Cambridge_Advanced/Mathematics_9709/9709_m19_qp_32.pdf}
% \cmdRef{10}{../../../Education/ExamPaper/Cambridge_Ages_16_19_Cambridge_Advanced/Mathematics_9709/9709_m19_qp_32.pdf}
% \cmdRef{12-13}{../../../Education/ExamPaper/Cambridge_Ages_16_19_Cambridge_Advanced/Mathematics_9709/9709_m19_qp_32.pdf}
% \cmdRef{14-15}{../../../Education/ExamPaper/Cambridge_Ages_16_19_Cambridge_Advanced/Mathematics_9709/9709_m19_qp_32.pdf}
% \cmdRef{16-17}{../../../Education/ExamPaper/Cambridge_Ages_16_19_Cambridge_Advanced/Mathematics_9709/9709_m19_qp_32.pdf}
% \cmdRef{2}{../../../Education/ExamPaper/Cambridge_Ages_16_19_Cambridge_Advanced/Mathematics_9709/9709_w18_qp_33.pdf}



% \pdfbookmark[0]{Pure Mathematics 1}{Pure Mathematics 1} 	\pdfbookmark[1]{Quadratics}{Quadratics} 		 	
% \cmdRef{2}{../../../Education/ExamPaper/Cambridge_Ages_16_19_Cambridge_Advanced/Mathematics_9709/9709_w18_qp_11.pdf}
% \cmdRef{3}{../../../Education/ExamPaper/Cambridge_Ages_16_19_Cambridge_Advanced/Mathematics_9709/9709_w18_qp_11.pdf}
% \cmdRef{2}{../../../Education/ExamPaper/Cambridge_Ages_16_19_Cambridge_Advanced/Mathematics_9709/9709_s18_qp_13.pdf}
