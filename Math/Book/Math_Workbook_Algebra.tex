\chapterimage{chapter_head_2.pdf}
\chapter{Algebra}


    \begin{tasks}[after-item-skip=2pt,after-skip=3pt, label-width=4ex](1)
         \task  Simplify $ 4(3x+5)-3(4x-5)$   \begin{envAnswer}[blankline=5]    $         $ \end{envAnswer}  
         \task  $ 4(3x+5)-2(2x-5)$       \begin{envAnswer}[blankline=4]    $         $ \end{envAnswer}  
         \task  $ 4(3x+5)-2(2x-5)$       \begin{envAnswer}[blankline=4]    $         $ \end{envAnswer}  
         \task  $ 4(3x+5)-2(2x-5)$       \begin{envAnswer}[blankline=4]    $         $ \end{envAnswer}  
         \task  $ 4(3x+5)-2(2x-5)$       \begin{envAnswer}[blankline=4]    $         $ \end{envAnswer}  
         \task  $ 4(3x+5)-2(2x-5)$       \begin{envAnswer}[blankline=4]    $         $ \end{envAnswer}  
    \end{tasks}

























\begin{enumerate}%%%[leftmargin=0cm] 

\item Simplify these expressions by adding or subtracting like terms.

\begin{tasks}[label=(\alph*), after-item-skip=2pt,after-skip=3pt, label-width=4ex](2)
    \task  $ 3x+4x=$                           \begin{envFillIn}$  -2     $    \end{envFillIn}
    \task  $ 4a+3a-2b-b$                       \begin{envFillIn}$   5     $    \end{envFillIn}
    \task  $ 4a-3b+7a-2b$                      \begin{envFillIn}$  -8     $    \end{envFillIn}
    \task  $ 4a-3b+7a-2b$                      \begin{envFillIn}$  -8     $    \end{envFillIn}
\end{tasks}

\item Simplify these expressions by adding or subtracting like terms.
\begin{tasks}[label=(\alph*), after-item-skip=2pt,after-skip=3pt, label-width=4ex](2)
    \task  $ 3x^3+4x^3=   $                    \begin{envFillIn}$  -2     $    \end{envFillIn}
    \task  $ 4x^2+10x^2=  $                    \begin{envFillIn}$   5     $    \end{envFillIn}
    \task  $ 5ab-4ab-2b+7b=$                         \begin{envFillIn}$  -8     $    \end{envFillIn}
\end{tasks}

\item Simplify these expressions by adding or subtracting like terms.
\begin{tasks}[label=(\alph*), after-item-skip=2pt,after-skip=3pt, label-width=4ex](2)
    \task  $ \frac{x^10}{x^2}         $                    \begin{envFillIn}$  -2     $    \end{envFillIn}
    \task  $ \frac{x^10}{x^5}         $                    \begin{envFillIn}$  -2     $    \end{envFillIn}
    \task  $ \frac{6x^6}{2x^2}        $                    \begin{envFillIn}$  -2     $    \end{envFillIn}
    \task  $ \frac{2x^2}{6x^6}        $                    \begin{envFillIn}$  -2     $    \end{envFillIn}
\end{tasks}


\end{enumerate}



\clearpage




\clearpage



\begin{bxTip}{Skill: Substitution}


If $a=-10$, find the value of $a^2$


\tcblower

$a^2=(-10)^2=100$ 

\tcblower

\begin{enumerate}[leftmargin=0cm] 

\item If $a=2$ and $b=5$, find the value of each expression.

\begin{tasks}[label=(\alph*), after-item-skip=2pt,after-skip=3pt, label-width=4ex](2)
    \task  $ 3a+4b=$                           \begin{envFillIn}$  -2     $    \end{envFillIn}
    \task  $ \frac{10a}{b}=$                   \begin{envFillIn}$   5     $    \end{envFillIn}
    \task  $ 4a(b-a)=$                         \begin{envFillIn}$  -8     $    \end{envFillIn}
\end{tasks}



\end{enumerate}

\end{bxTip}


\clearpage

\begin{enumerate}[leftmargin=0cm] 

\item Expand and simplify.
    \begin{tasks}[label=(\alph*), after-item-skip=2pt,after-skip=3pt, label-width=4ex](2)
         \task  $ 4(3x+5)-3(4x-5)$       \begin{envAnswer}[blankline=2]    $         $ \end{envAnswer}  
         \task  $ 4(3x+5)-2(2x-5)$       \begin{envAnswer}[blankline=2]    $         $ \end{envAnswer}  
    \end{tasks}





\end{enumerate}



\begin{enumerate}%[leftmargin=0cm] 

\item 7123456 $x^2+8x$ can be written in the form $(x+a)^2+b$. Find the value of $a$ and the value of $b$.
    \begin{envAnswer}[blankline=3]         $     4          $ \end{envAnswer}
    \begin{envAnswer}[blankline=5]         $   -16          $ \end{envAnswer}


\item $x^2+7x -3$ can be written in the form $(x+a)^2+b$. Find the value of $a$ and the value of $b$.
    \begin{envAnswer}[blankline=3]         $     3.5        $ \end{envAnswer}
    \begin{envAnswer}[blankline=5]         $   -15.25       $ \end{envAnswer}


\item $x^2-6x -10$ is to be written in the form $(x-p)^2+q$. Find the value of $p$ and the value of $q$.
    \begin{envAnswer}[blankline=3]         $     3.5        $ \end{envAnswer}
    \begin{envAnswer}[blankline=0]         $   -15.25       $ \end{envAnswer}


\item $2x^2+5x -10$ is to be written in the form $2(x+p)^2+q$. Find the value of $p$ and the value of $q$.
    \begin{envAnswer}[blankline=3]         $     3.5        $ \end{envAnswer}
    \begin{envAnswer}[blankline=0]         $   -15.25       $ \end{envAnswer}


\item The equation $5-\frac{3}{x^2}=-5x-2$ can be written in the form $x^3+ax^2+b=0$. Find the value of $a$ and the value of $b$.
    \begin{envAnswer}[blankline=3]         $     3.5        $ \end{envAnswer}
    \begin{envAnswer}[blankline=0]         $   -15.25       $ \end{envAnswer}




\end{enumerate}

\clearpage

\begin{enumerate} [leftmargin=0cm] 

\item Write down the next 3 terms for each sequence:
\begin{tasks}[label=(\alph*), after-item-skip=2pt,after-skip=3pt, label-width=4ex](1)
    \task  $ 5,  \quad 8,  \quad 11, \quad 14, \quad           $                                  \begin{envFillIn} $   17, \quad 21                         $ \end{envFillIn}
    \task  $ 5,  \quad 11,  \quad 17, \quad 23, \quad          $                                  \begin{envFillIn} $   30, \quad 37                         $ \end{envFillIn}
    \task  $ 1,  \quad 3,  \quad 6, \quad 10, \quad 15, \quad  $                                  \begin{envFillIn} $   21, \quad 27                         $ \end{envFillIn}
    \task  $ 61, \quad 58, \quad 55, \quad 52, \quad           $                                  \begin{envFillIn} $   49, \quad 46                         $ \end{envFillIn}
    \task  $ 1 , \quad 4,  \quad 9, \quad 16,  \quad           $                                  \begin{envFillIn} $   25, \quad 36                         $ \end{envFillIn}
    \task  $ \frac{1}{2} , \quad \frac{3}{4},  \quad \frac{5}{8}, \quad \frac{7}{16},  \quad   $  \begin{envFillIn} $   \frac{9}{32}, \quad \frac{11}{64}    $ \end{envFillIn}
    \task  $ x-y , \quad x-3y,  \quad x-5y, \quad x-7y,  \quad   $                                \begin{envFillIn} $   x-9y, \quad x-11y                    $ \end{envFillIn}
    \task  $ 1,  \quad 2,  \quad 3, \quad 5, \quad 8, \quad 11, \quad  $                          \begin{envFillIn} $   19, \quad 30                         $ \end{envFillIn}

\end{tasks}


\item A sequence begins
      \begin{align*}
           14  \quad 17   \quad  20  \quad  23  \quad  26  \quad  29
      \end{align*}       
    Write down a formula for the $n$th term of this sequence.       

    \begin{envAnswer}[blankline=2,text=] $ 3n+11      $ \end{envAnswer}

\item A sequence begins \\
      \begin{align*}
            30  \quad 32.5   \quad  35  \quad  37.5  \quad  40  \quad  42.5
      \end{align*}       
    Write down a formula for the $n$th term of this sequence.       

    \begin{envAnswer}[blankline=2,text=] $ 2.5n+27.5      $ \end{envAnswer}
    
\item A sequence begins \\
      \begin{align*}
            60  \quad 53  \quad  46  \quad  39  \quad  32  \quad  25
      \end{align*}       
    Write down a formula for the $n$th term of this sequence.       

    \begin{envAnswer}[blankline=2,text=] $ 67-7n      $ \end{envAnswer}
    

\item A sequence begins \\
      \begin{align*}
            1  \quad 7  \quad  17  \quad  31  \quad  49  \quad  71 
      \end{align*}       
    Write down a formula for the $n$th term of this sequence.       

    \begin{envAnswer}[blankline=2,text=] $ 67-7n      $ \end{envAnswer}

\item A sequence begins \\
      \begin{align*}
            7  \quad  17  \quad  31  \quad  49  \quad  71 
      \end{align*}       
    Write down a formula for the $n$th term of this sequence.       

    \begin{envAnswer}[blankline=2,text=] $ 67-7n      $ \end{envAnswer}


\item A sequence begins \\
      \begin{align*}
            2  \quad  6  \quad  12  \quad  20  \quad  30 
      \end{align*}       
    Write down a formula for the $n$th term of this sequence.       

    \begin{envAnswer}[blankline=2,text=] $ 67-7n      $ \end{envAnswer}

    
\item A sequence begins \\
      \begin{align*}
            1  \quad  5  \quad  11  \quad  19  \quad  29 
      \end{align*}       
    Write down a formula for the $n$th term of this sequence.       

    \begin{envAnswer}[blankline=2,text=] $ 67-7n      $ \end{envAnswer}




\item A sequence begins \\
      \begin{align*}
            11  \quad  15  \quad  21  \quad  29  \quad  39 
      \end{align*}       
    Write down a formula for the $n$th term of this sequence.       

    \begin{envAnswer}[blankline=2,text=] $ 67-7n      $ \end{envAnswer}








\item A sequence begins \\
      \begin{align*}
            2  \quad  9  \quad  28  \quad  65  \quad  126 
      \end{align*}       
    Write down a formula for the $n$th term of this sequence.       

    \begin{envAnswer}[blankline=2,text=] $ 67-7n      $ \end{envAnswer}


\item A sequence begins \\
      \begin{align*}
            0  \quad  7  \quad  26  \quad  63  \quad  124 
      \end{align*}       
    Write down a formula for the $n$th term of this sequence.       

    \begin{envAnswer}[blankline=2,text=] $ 67-7n      $ \end{envAnswer}
    
\end{enumerate}




\clearpage

\begin{enumerate} [leftmargin=0cm] 
\item Expand and simplify.
\begin{tasks}[label=(\alph*), after-item-skip=2pt,after-skip=3pt, label-width=4ex](2)
    \task  $ (x+ 9)^2   $    \\  \begin{envAnswer} $   x^2 +   18x +    81       $ \end{envAnswer}
    \task  $ (x+10)^2   $    \\  \begin{envAnswer} $   x^2 +   20x +   100       $ \end{envAnswer}
    \task  $ (x+ 5)^2   $    \\  \begin{envAnswer} $   x^2 +   10x +    25       $ \end{envAnswer}
    \task  $ (x+12)^2   $    \\  \begin{envAnswer} $   x^2 +   24x +   144       $ \end{envAnswer}
\end{tasks}

\item Expand and simplify.
\begin{tasks}[label=(\alph*), after-item-skip=2pt,after-skip=3pt, label-width=4ex](2)
    \task  $ (2x+ 9)^2  $    \\  \begin{envAnswer} $  4x^2 +   36x +    81       $ \end{envAnswer}
    \task  $ (2x+10)^2  $    \\  \begin{envAnswer} $  4x^2 +   40x +   100       $ \end{envAnswer}
    \task  $ (3x+ 5)^2  $    \\  \begin{envAnswer} $  9x^2 +   30x +    25       $ \end{envAnswer}
    \task  $ (5x+12)^2  $    \\  \begin{envAnswer} $ 25x^2 +  120x +   144       $ \end{envAnswer}
\end{tasks}


\begin{tasks}[label=(\alph*), after-item-skip=2pt,after-skip=3pt, label-width=4ex](2)
    \task  $ (a-30)^2   $    \\  \begin{envAnswer} $   a^2 -   60a +   900       $ \end{envAnswer}
    \task  $ (x- 6)^2   $    \\  \begin{envAnswer} $   x^2 -   12x +    36       $ \end{envAnswer}
    \task  $ (m- 5)^2   $    \\  \begin{envAnswer} $   m^2 -   10m +    25       $ \end{envAnswer}
    \task  $ (n- 1)^2   $    \\  \begin{envAnswer} $   n^2 -    2n +     1       $ \end{envAnswer}
\end{tasks}


\item Expand and simplify.
\begin{tasks}[label=(\alph*), after-item-skip=2pt,after-skip=3pt, label-width=4ex](2)
    \task  $ ( a+ 5) (a-5)                 $    \\  \begin{envAnswer} $   a^2   -   25        $ \end{envAnswer}
    \task  $ ( b+ 6) (b-6)                 $    \\  \begin{envAnswer} $   b^2   -   36        $ \end{envAnswer}
    \task  $ ( x-10) (x+10)                $    \\  \begin{envAnswer} $   x^2   -  100        $ \end{envAnswer}
    \task  $ ( m-15) (m+15)                $    \\  \begin{envAnswer} $   m^2   -  225        $ \end{envAnswer}
\end{tasks}



\item Expand and simplify.
\begin{tasks}[label=(\alph*), after-item-skip=2pt,after-skip=3pt, label-width=4ex](2)
    \task  $ ( x+ 5y)^2  $    \\  \begin{envAnswer} $   x^2 +  10xy +  25y^2      $ \end{envAnswer}
    \task  $ (5x+ 6y)^2  $    \\  \begin{envAnswer} $  4x^2 +  60xy +  36y^2      $ \end{envAnswer}
    \task  $ (3x+16y)^2  $    \\  \begin{envAnswer} $  9x^2 +  96xy + 256y^2      $ \end{envAnswer}
    \task  $ (5x+15y)^2  $    \\  \begin{envAnswer} $ 25x^2 +  150x + 255y^2      $ \end{envAnswer}
\end{tasks}


\item Expand and simplify.
\begin{tasks}[label=(\alph*), after-item-skip=2pt,after-skip=3pt, label-width=4ex](2)
    \task  $ (5m+ 2n)^2  $    \\  \begin{envAnswer} $ 25m^2 +  20mn +   4n^2      $ \end{envAnswer}
    \task  $ (8a+ 7b)^2  $    \\  \begin{envAnswer} $ 64a^2 + 112ab +  49b^2      $ \end{envAnswer}
    \task  $ (3p+ 7q)^2  $    \\  \begin{envAnswer} $  9p^2 +  42pq +  49q^2      $ \end{envAnswer}
    \task  $ (5c+ 9d)^2  $    \\  \begin{envAnswer} $ 25c^2 +  90cd +  81d^2      $ \end{envAnswer}
\end{tasks}





\item Expand and simplify.
\begin{tasks}[label=(\alph*), after-item-skip=2pt,after-skip=3pt, label-width=4ex](2)
    \task  $ (7a- 3b)^2  $    \\  \begin{envAnswer} $ 49a^2 -  42ab +   9b^2      $ \end{envAnswer}
    \task  $ (5x- 6y)^2  $    \\  \begin{envAnswer} $ 25x^2 -  60xy +  36y^2      $ \end{envAnswer}
    \task  $ (3x-11y)^2  $    \\  \begin{envAnswer} $  9x^2 -  66xy + 121y^2      $ \end{envAnswer}
    \task  $ (5m+13n)^2  $    \\  \begin{envAnswer} $ 25m^2 - 130mn + 169n^2      $ \end{envAnswer}
\end{tasks}


\item Expand and simplify.
\begin{tasks}[label=(\alph*), after-item-skip=2pt,after-skip=3pt, label-width=4ex](2)
    \task  $ (5a- 2b)^2  $    \\  \begin{envAnswer} $ 25a^2 -  20ab +   4b^2      $ \end{envAnswer}
    \task  $ (8m- 7n)^2  $    \\  \begin{envAnswer} $ 64m^2 - 112mn +  49n^2      $ \end{envAnswer}
    \task  $ (3c- 8d)^2  $    \\  \begin{envAnswer} $  9c^2 -  48cd +  64d^2      $ \end{envAnswer}
    \task  $ (5x- 9y)^2  $    \\  \begin{envAnswer} $ 25x^2 -  90xy +  81y^2      $ \end{envAnswer}
\end{tasks}




\item Expand and simplify.
\begin{tasks}[label=(\alph*), after-item-skip=2pt,after-skip=3pt, label-width=4ex](2)
    \task  $ ( 8+ a) (a-8)                 $    \\  \begin{envAnswer} $   a^2   -   64        $ \end{envAnswer}
    \task  $ ( b+ 6) (6-b)                 $    \\  \begin{envAnswer} $   36    -   b^2       $ \end{envAnswer}
    \task  $ (10-x ) (x+10)                $    \\  \begin{envAnswer} $   100   -   x^2       $ \end{envAnswer}
    \task  $ ( m-20) (20+m)                $    \\  \begin{envAnswer} $   m^2   -   400       $ \end{envAnswer}

\end{tasks}


\item Show that $\frac{(x-8)^2}{4}-25  $ is same as $\frac{(x-18)(x+2)}{4}$



\clearpage    
\item Factorise these quadratic expressions.
\begin{tasks}[label=(\alph*), after-item-skip=2pt,after-skip=3pt, label-width=4ex](2)
    \task  $ x^2 -5x                               $  \\  \begin{envAnswer} $  x(x-5)            $ \end{envAnswer}
    \task  $ x^2 -5x +6                            $  \\  \begin{envAnswer} $  (x-2)(x-3)        $ \end{envAnswer}
    \task  $ x^2 -5x -6                            $  \\  \begin{envAnswer} $  (x+1)(x-6)        $ \end{envAnswer}
    \task  $ x^2 +5x -6                            $  \\  \begin{envAnswer} $  (x-1)(x+6)        $ \end{envAnswer}
    \task  $ x^2 -6x +8                            $  \\  \begin{envAnswer} $  (x-2)(x-4)        $ \end{envAnswer}
    \task  $ x^2 +6x +8                            $  \\  \begin{envAnswer} $  (x+2)(x+4)        $ \end{envAnswer}
    \task  $ x^2 +8x +15                           $  \\  \begin{envAnswer} $  (x+3)(x+5)        $ \end{envAnswer}
    \task  $ x^2 -4x -32                           $  \\  \begin{envAnswer} $  (x+4)(x-8)        $ \end{envAnswer}
    \task  $ x^2 -16                               $  \\  \begin{envAnswer} $  (x+4)(x-4)        $ \end{envAnswer}
    \task  $ 4x^2 -16                              $  \\  \begin{envAnswer} $  4(x+2)(x-2)       $ \end{envAnswer}
    \task  $ 2x^2 -50                              $  \\  \begin{envAnswer} $  2(x+5)(x-5)       $ \end{envAnswer}
\end{tasks}    


\item Factorise these quadratic expressions.
	\begin{tasks}[label=(\alph*), after-item-skip=2pt,after-skip=3pt, label-width=4ex](2)
		\task  $ x^2 -16y^2                            $  \\  \begin{envAnswer} $  (x+4y)(x-4y)      $ \end{envAnswer}
		\task  $ 25x^2 -9y^2                           $  \\  \begin{envAnswer} $  (5x+3y)(5x-3y)    $ \end{envAnswer}
	\end{tasks}    


\item Factorise the following by grouping.
    \begin{tasks}[label=(\alph*), after-item-skip=2pt,after-skip=3pt, label-width=4ex](2)
	    \task $a+b+ac+bc$             \\ \begin{envAnswer}[blankline=3]         $                     $ \end{envAnswer}
		\task $1-x-y+xy$              \\ \begin{envAnswer}[blankline=3]         $(1-x)(1-y)           $ \end{envAnswer} 
		\task $3ac-4ad-6bc+8bd$       \\ \begin{envAnswer}[blankline=3]         $(a-2b)(3c-4d)        $ \end{envAnswer}
        \task $xy-4x-6y+24$           \\ \begin{envAnswer}[blankline=3]         $                     $ \end{envAnswer}		
		\task $x^2-2x+6y-3xy$         \\ \begin{envAnswer}[blankline=3]         $(x-2)(x-3y)          $ \end{envAnswer}	
		\task $a^2-2ab-3ac+6bc$       \\ \begin{envAnswer}[blankline=3]         $(a-2b)(a-3c)          $ \end{envAnswer}
    \end{tasks}

    
\item Write as a single fraction.
\begin{tasks}[label=(\alph*), after-item-skip=2pt,after-skip=3pt, label-width=4ex](2)
    \task  $ \frac{1}{x}-x                          $    
                                                         % \\  \begin{envAnswer} $   \frac{1-x^2}{x}           $ \end{envAnswer}
    \task  $ \frac{1}{x}-\frac{x}{2}                $    \\  \begin{envAnswer} $   \frac{2-x^2}{2x}          $ \end{envAnswer}
    \task  $ \frac{1}{x+1}-\frac{x-2}{x^2+x}        $    \\  \begin{envAnswer} $   \frac{2}{x(x+1)}          $ \end{envAnswer}
    \task  $ \frac{1}{x+1}-\frac{x}{x^2-1}          $    \\
    
             \begin{envAnswer} $ \displaystyle  -\frac{1}{(x+1)(x-1)}         $ \end{envAnswer}
    
    
    
    \task  $ \frac{3}{2(x+7)}-\frac{x+8}{x^2-49}    $   \\   
            %\begin{envWorkedSolL} [0.9\textwidth]{ $   =\frac{3}{2(x+7)}-\frac{x+8}{(x+7)(x-7)}                          $ }\end{envWorkedSolL}    \\
            %\begin{envWorkedSolL} [\textwidth]{ $   = \frac{3(x-7)}{2(x+7)(x-7)}-\frac{2(x+8)}{2(x+7)(x-7)}           $ }\end{envWorkedSolL}    \\
            % \begin{envWorkedSolL} [\textwidth]{ $   = \frac{3(x-7)-2(x+8)}{2(x+7)(x-7)}                               $ }\end{envWorkedSolL} 
	   % \begin{envSolution}
             % \begin{envWorkedSolL} [\textwidth]{ $   =\frac{3}{2(x+7)}-\frac{x+8}{(x+7)(x-7)}                          $ }\end{envWorkedSolL}    \\
             % \begin{envWorkedSolL} [\textwidth]{ $   = \frac{3(x-7)}{2(x+7)(x-7)}-\frac{2(x+8)}{2(x+7)(x-7)}           $ }\end{envWorkedSolL}    \\
            % \begin{envWorkedSolL} [0.5\textwidth]{ $   = \frac{3(x-7)-2(x+8)}{2(x+7)(x-7)}                               $ }\end{envWorkedSolL} 
        % \end{envSolution}


    \task  $ \frac{3}{2(x+7)}-\frac{x+8}{x^2-49}    $   \\   
    \task  $ \frac{3}{2(x+7)}-\frac{x+8}{x^2-49}    $   \\   
 		
            % \begin{envWorkedSolL} [0.5\textwidth]{ $   = \frac{3x-21-2x-16}{2(x+7)(x-7)}                                 $ }\end{envWorkedSolL}    \\
            % \begin{envWorkedSolS} [0.5\textwidth]{ $   = \frac{x-37}{2(x+7)(x-7)}                                        $ }\end{envWorkedSolS}    \\            
            \begin{bxSolution}
             \begin{align*}
             =\frac{3}{2(x+7)}-\frac{x+8}{(x+7)(x-7)}               \\
             = \frac{3(x-7)}{2(x+7)(x-7)}-\frac{2(x+8)}{2(x+7)(x-7)}  \\     = \frac{3(x-7)-2(x+8)}{2(x+7)(x-7)}            \\ 
             = \frac{3x-21-2x-16}{2(x+7)(x-7)}   \\                             
             = \frac{x-37}{2(x+7)(x-7)}                                  \end{align*}      
             \end{bxSolution}
            % \\
            \begin{envAnswer}       $  \frac{x-37}{2(x+7)(x-7)}                                         $ \end{envAnswer}


\end{tasks}


\end{enumerate}







% \clearpage


% \begin{enumerate} [leftmargin=0cm] 
    % \item Solve the following equations.\\
        % \begin{tasks}[label=(\alph*), after-item-skip=2pt,after-skip=3pt, label-width=4ex](2)
        % \task  $ (x-5)(2x+5)=0                    $  \begin{envAnswer} $ x=5             $    or $ x=-2.5          $\end{envAnswer}  
        % \task  $ x(3x-6)=0                        $  \begin{envAnswer} $ x=0             $    or $ x=2             $\end{envAnswer}    
        % \task  $ (3x-1)(x+7)=0                    $  \begin{envAnswer} $ x=\frac{1}{3}   $    or $ x=-7            $\end{envAnswer}    
        % \task  $ (4x-1)(x-2)=0                    $  \begin{envAnswer} $ x=0.25          $    or $ x=2             $\end{envAnswer}    
        % \task  $ (x+1)(x+2)=0                     $  \begin{envAnswer} $ x=-1            $    or $ x=-2            $\end{envAnswer}    
        % \task  $ (x-10)(x-12)=0 $         
               % \begin{envSolution} 
                    % \noindent Solution:\\[15pt]  
                    % $ 1+1  $           \\
					   % \begin{envAnswer} $ x=10            $    or $ x=12            $\end{envAnswer}   
               % \end{envSolution}
			   
        % \end{tasks}
% \end{enumerate}

% \begin{enumerate} [leftmargin=0cm] 
    % \item Solve the equation $ \frac{1}{x-5} - \frac{2}{2x+5} = \frac{3}{3x-5} $. \\
        % \begin{envSolution} 
             % %-----------------------------------------------------------------------------------------------
             % % Note: Need to add a blank line before math equation, otherwise showing below error message:
             % %               ! Missing $ inserted.
             % %                <inserted text>
             % % Date:
             % %      2020/07/31
             % %             
             % % -----------------------------------------------------------------------------------------------
        % \noindent Solution:\\[15pt]     
             % $ \frac{2x+5}{(x-5)(2x+5)}-\frac{2(x-5)}{(x-5)(2x+5)} =\frac{3}{3x-5}    $ \\[20pt]
             % $ \frac{2x+5-2(x-5)}{(x-5)(2x+5)}=\frac{3}{3x-5}                         $ \\[20pt]
             % $\frac{2x+5-2x+10}{(x-5)(2x+5)}=\frac{3}{3x-5}                           $ \\[20pt]
             % $\frac{15}{(x-5)(2x+5)}=\frac{3}{3x-5}                                   $ \\[20pt]
             % $3(x-5)(2x+5)=15(3x-5)                                                   $ \\[10pt]
             % $(x-5)(2x+5)=5(3x-5)                                                     $ \\[10pt]
             % $2x^2-5x-25=15x-25                                                       $ \\[10pt]
             % $2x^2-20x=0                                                              $ \\[10pt]
             % $x^2-10x=0                                                               $ \\[10pt]
             % $ x=0 $ or $ x= 10                                                       $ \\
			 
			 
             % \begin{envAnswer}[blankline=10] $   x=0 or 10      $ \end{envAnswer}
        % \end{envSolution}
        

% \item aaa aa \\

% \end{enumerate}


% \clearpage

% \begin{enumerate} [leftmargin=0cm] 

% \item Solve, for real values of $x$ , the inequality
           % \[  \lvert 2x-10 \rvert \leqslant x+5  \]

 

% \begin{envAnswer}[blankline=5] $                                                           $ \end{envAnswer}


% \item Solve, for real values of $x$ , the inequality
           % \[ \frac{3x+4}{2-x} \leqslant 7  \]

 

% \begin{envAnswer}[blankline=5] $                                                      1     $ \end{envAnswer}


% \item Showing all your working, find the square root(s) of the complex number $z = 3 - 4i$.
% \begin{envAnswer}[blankline=5] $                                                           $ \end{envAnswer}

% \item Let $F_n(x)=\int cos^nx \; dx$. \\
      % By rewriting $cos^nx $ as $cosx \; cos^{n-1}x $ or otherwise, prove that \\
      % \[ F_n(x)=\frac{1}{n} cos^{n-1}x \; sinx + \left (\frac{n-1}{n} \right ) F_{n-2}(x) \] \\

            % \begin{envAnswer}[blankline=6] $                                                           $ \end{envAnswer}

% \item Find all solutions $ (3+2\sqrt{2})^{x^2-4x+3} + (3-2\sqrt{2})^{x^2-4x+3} = 6$, expressing your answer in exact form.\\
          % %	  \begin{envWorkedSolL}{ $  aa1 qqq                                                    $ }\end{envWorkedSolL}    \\
            % % \begin{envWorkedSolL}{ $                aab2                                                     $ }\end{envWorkedSolL}    \\
            % % \begin{envWorkedSolL}{ $              aa3                                                       $ }\end{envWorkedSolL}    \\             
            % % \begin{envWorkedSolL}{ $                a444                                                     $ }\end{envWorkedSolL}    \\
            % % \begin{envWorkedSolS}[0.5\textwidth] { $                  a555                                                   $ }\end{envWorkedSolS}    \\            
            % \begin{envAnswer}[blankline=10] $             a6                                                          $ \end{envAnswer}

% \end{enumerate}


\clearpage



\begin{bxExample}{Example}    
    \textbf{Example}
	\begin{itemize}
	     \item[] Solve the equation $cos^{2}x=\frac{1}{4} $ for $0\degree \leqslant x \leqslant 360\degree. $
	\end{itemize}
\tcbline 

Solution:\indent \indent \indent \indent
\vspace{0.5cm} 

$\begin{WithArrows}%[format = l]
    \qquad & \cos^{2}x=\frac{1}{4}   &                  a        &       a                   &     a\\[10pt] 
	       & \cos x=\pm \frac{1}{2}   &                          &                          &      \\[10pt]
	       & x=\cos^{-1}(\frac{1}{2})&                          & x=\cos^{-1}(-\frac{1}{2}) &       \\[10pt]
	       & x=60\degree             & x=-60\degree             & x=120\degree             & x=-120\degree     \\[10pt]
           & x=60\degree             & x=-60\degree +360\degree & x=120\degree             & x=-120\degree + 360\degree    \\
           & x=60\degree             & x=300\degree             & x =120\degree            &  x=240\degree
\end{WithArrows}$

\tcbline


Note:
\begin{itemize}
       % \item[] Above is one common error in solving absolute value inequlities as 
                % \[    \lvert a \rvert \geqslant b       \centernot\implies     a^2 \geqslant b^2  \quad  \quad \text{if b is negative.} \]    
    
      \item[]  For example, $x=0$ is a valid value in the above inequality but
                % \[     \lvert 0 \rvert \geqslant -10       \centernot\implies     0^2 \geqslant (-10)^2 \] 
\end{itemize}

\end{bxExample}


