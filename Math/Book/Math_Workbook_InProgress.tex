\newtcbox{envDashBox}{enhanced,frame hidden,borderline={0.5pt}{0pt}{red,dashed} }


\begin{envDashBox} mark 123
     $\frac{1}{x}  $
\end{envDashBox}
\clearpage

\begin{enumerate}[leftmargin=0cm] 

\item The curve of $f(x)=x^3-ax^2+bx+4 $ has a tangent with gradient 2 at the point (1,6).\\Find the coordinates of the point on the curve where $x$=2.
    \begin{envAnswer}[blankline=8]         $   f(x)=x^3-2x^2+3x+4            $ \end{envAnswer}


\end{enumerate}



\clearpage



\begin{bxTip}{Skill: Composite functions}
\begin{enumerate}[leftmargin=0cm] 

\item Let $f(x)=(x+1)^3$. Let $g$ be a function so that $fg(x)=27x^9$. Find $g(x)$.
    \begin{envAnswer}[blankline=3]         $               $ \end{envAnswer}


\item Let $f(x)=(x+1)^2$. Let $g$ be a function so that $fg(x)=8x^6$. Find $g(x)$.
    \begin{envAnswer}[blankline=3]         $               $ \end{envAnswer}

\item Let $g(x)=x+5$. Let $g$ be a function so that $fg(x)=x^2+10x+25$. Find $f(x)$.
    \begin{envAnswer}[blankline=3]         $               $ \end{envAnswer}


\end{enumerate}

\end{bxTip}


\begin{bxTip}{Skill: Integration by reverse of chain rule}
    \pdfbookmark[section]{Integration}{Reverse of chain rule}

    Find $ \int x^2 cos(x^3-5) \, dx         $

\tcbline 

Analysis and elaboration:\indent \indent \indent \indent
\vspace{0.5cm} 

$\begin{WithArrows}
       \qquad   &  \int x^2 cos(x^3-5)  \, dx 	                   \Arrow{\circled{1} Observe: $(x^3-5)=3x^2 $ }   \\[20pt]
	          = &  \int \frac{1}{3}(x^3-5)' \cos(x^3-5) dx         \Arrow{\circled{2}                          }\\[20pt]
              = &  \frac{1}{3} \int  (x^3-5)' \cos(x^3-5) dx                  \\[10pt]                                  
              = &                                                  \Arrow{\circled{3} \HandRight Integration by reverse of chain rule. } \\[10pt]     
			  = &  \frac{1}{3} \sin(x^3-5)+c
\end{WithArrows}$


\tcbline

Solution:\indent \indent \indent \indent
\vspace{0.5cm} 

$\begin{WithArrows}
       \qquad   &  \int xe^{2x} \, dx 	                             \\[20pt]
	          = & \int x \cdot (\frac{1}{2} e^{2x})' \, dx           \\[20pt]
              = & x \cdot \frac{1}{2} e^{2x} -    \int (x)' \cdot \frac{1}{2} e^{2x} \, dx     \\[10pt]                                  
              = & x \cdot \frac{1}{2} e^{2x} -    \int 1 \times \frac{1}{2} e^{2x} \, dx       \\[10pt]                                  
              = & \frac{x}{2} e^{2x} -    \int \frac{1}{2} e^{2x} \, dx                \\[10pt]     
              = & \frac{x}{2} e^{2x} -    \frac{1}{4} e^{2x} + c                       \\[10pt]     
			  = & \frac{1}{4} (2x-1)e^{2x}+c
\end{WithArrows}$


\end{bxTip}




\clearpage



\begin{bxTip}{Skill: Integration by parts}
    \pdfbookmark[section]{Integration}{Integration by parts}

    Find $ \int xe^{2x} \, dx         $

\tcbline 

Analysis and elaboration:\indent \indent \indent \indent
\vspace{0.5cm} 

$\begin{WithArrows}
       \qquad   &  \int xe^{2x} \, dx 	                           \Arrow{\circled{1} Integrate $e^{2x}$ }   \\[20pt]
	          = & \int x \cdot (\frac{1}{2} e^{2x})' \, dx         \Arrow{\circled{2} \HandRight Integration by parts: $\int uv' \, dx  = uv - \int u'v \, dx $   }  \\[20pt]
              = & x \cdot \frac{1}{2} e^{2x} -    \int (x)' \cdot \frac{1}{2} e^{2x} \, dx     \\[10pt]                                  
              = & x \cdot \frac{1}{2} e^{2x} -    \int 1 \times \frac{1}{2} e^{2x} \, dx       \\[10pt]                                  
              = & \frac{x}{2} e^{2x} -    \int \frac{1}{2} e^{2x} \, dx                \\[10pt]     
              = & \frac{x}{2} e^{2x} -    \frac{1}{4} e^{2x} + c                       \Arrow{\circled{3} Factorising }   \\[10pt]     
			  = & \frac{1}{4} (2x-1)e^{2x}+c
\end{WithArrows}$


\tcbline

Solution:\indent \indent \indent \indent
\vspace{0.5cm} 

$\begin{WithArrows}
       \qquad   &  \int xe^{2x} \, dx 	                             \\[20pt]
	          = & \int x \cdot (\frac{1}{2} e^{2x})' \, dx           \\[20pt]
              = & x \cdot \frac{1}{2} e^{2x} -    \int (x)' \cdot \frac{1}{2} e^{2x} \, dx     \\[10pt]                                  
              = & x \cdot \frac{1}{2} e^{2x} -    \int 1 \times \frac{1}{2} e^{2x} \, dx       \\[10pt]                                  
              = & x \cdot \frac{1}{2} e^{2x} -    \int \frac{1}{2} e^{2x} \, dx                \\[10pt]     
              = & x \cdot \frac{1}{2} e^{2x} -    \frac{1}{4} e^{2x} + c                       \\[10pt]     
			  = & \frac{1}{4} (2x-1)e^{2x}+c
\end{WithArrows}$


\end{bxTip}




\clearpage



\begin{bxExample}{Example}    
    \textbf{Example}
	\begin{itemize}
	     \item[] The equation $5-\frac{3}{x^2}=-5x-2$ can be written in the form $x^3+ax^2+b=0$.\\
	     Find the value of $a$ and the value of $b$.
	\end{itemize}
\tcbline 

Solution:\indent \indent \indent \indent
\vspace{0.5cm} 

$\begin{WithArrows}%[format = l]
       \qquad &  5-\frac{3}{x^2}  =-5x-2 	                 \Arrow{Multiply by $x^2$}\\[10pt]
	         & 5x^2 - 3        = -5x^3 -2x^2                 \Arrow{Rearranging} \\[10pt]
             & 5x^3+ 7x^2 - 3  = 0                                               
\end{WithArrows}$

\tcbline


Note:
\begin{itemize}
       % \item[] Above is one common error in solving absolute value inequlities as 
                % \[    \lvert a \rvert \geqslant b       \centernot\implies     a^2 \geqslant b^2  \quad  \quad \text{if b is negative.} \]    
    
      \item[]  For example, $x=0$ is a valid value in the above inequality but
                % \[     \lvert 0 \rvert \geqslant -10       \centernot\implies     0^2 \geqslant (-10)^2 \] 
\end{itemize}

\end{bxExample}


\begin{introduction}
    \item Definition of Theorem
	\item Ask for help
	\item Optimization Problem
	\item Property of Cauchy Series
	\item Angle of Corner
\end{introduction}



0000abcd1230000\\

\begin{tikzpicture}
  \begin{axis} [axis lines=center]
    \addplot [domain=-1.6:2.6, smooth, thick] { x^3 -3*x^2+2*x};
  \end{axis}
\end{tikzpicture}


\begin{taggedblock}{tgRefBook}
    \bxtgRefBook{Pure Mathematics 2 and 3 by Sophie Goldie, ISBN 978-1-144441-4646-2, page 124}
\end{taggedblock}


abcd123\\

\begin{envRefBook}
    Pure Mathematics 2 an
\end{envRefBook}


\begin{bxTip}[colbacktitle=bkcolor]{Skill: Rearranging Formulae}
    Rearrange the formula $\frac{1}{f}=\frac{1}{u}+\frac{1}{v} $ to make $f$ the subject of the formula. \\
    
	\tcbline
    Solution:    
    %\begin{bxExample}[ams align*]
    \begin{align*}
        \frac{1}{f}&=\frac{1}{u}+\frac{1}{v}         && \text{ }                              \\[10pt]
        \frac{1}{f}&=\frac{u}{uv}+\frac{v}{uv}       && \text{; Rearranging}                   \\[10pt]
        \frac{1}{f}&=\frac{u+v}{uv}                  && \text{; Write as a simple fraction}    \\[10pt]
        \frac{f}{1}&=\frac{uv}{u+v}                  && \text{ }                            \\[10pt]
        f&=\frac{uv}{u+v}                            && \text{;Reciprocal}  
\end{align*}        
    %\end{bxExample}
\end{bxTip}



\begin{bxTip}[colbacktitle=green]{Skill: Rearranging Formulae}
    Rearrange the formula $\frac{1}{f}=\frac{1}{u}+\frac{1}{v} $ to make $f$ the subject of the formula. \\
    
    Solution:    
    %\begin{bxExample}[ams align*]
    \begin{align*}
        \frac{1}{f}&=\frac{1}{u}+\frac{1}{v}         && \text{ }                              \\[10pt]
        \frac{1}{f}&=\frac{u}{uv}+\frac{v}{uv}       && \text{; Rearranging}                   \\[10pt]
        \frac{1}{f}&=\frac{u+v}{uv}                  && \text{; Write as a simple fraction}    \\[10pt]
        \frac{f}{1}&=\frac{uv}{u+v}                  && \text{ }                            \\[10pt]
        f&=\frac{uv}{u+v}                            && \text{;Reciprocal}  
\end{align*}        
    %\end{bxExample}
\end{bxTip}

\clearpage
\setcounter{page}{1}

\begin{bxExamHeader}
    
    \pdfbookmark[section]{Year 8 Algebra}{sec2}
    \begin{center}\huge{Mathematics and Statistics Year 8 Algebra} \end{center}
	\null\hfill Elite Tuition Centre \\
    Instructions 
    \begin{itemize}
        \item Without sufficient working, correct answers may be awarded no marks.
		\item Calculators must not be used.
        \item If the degree of accuracy is not specified in the question, and if the answer is not exact, give the answer to
three significant figures. Give answers in degrees to one decimal place.
    \end{itemize}
	
    \vspace*{\baselineskip}
	
	Advice
	\begin{itemize}
	    \item Read each question carefully before you start to answer it.
        \item Check your answers if you have time at the end.
	\end{itemize}
\end{bxExamHeader}


\begin{enumerate}%[leftmargin=0cm] 

\item $x^2+8x$ can be written in the form $(x+a)^2+b$. Find the value of $a$ and the value of $b$.
    \begin{envAnswer}[blankline=3]         $     4          $ \end{envAnswer}
    \begin{envAnswer}[blankline=0]         $   -16          $ \end{envAnswer}


\item $x^2+7x -3$ can be written in the form $(x+a)^2+b$. Find the value of $a$ and the value of $b$.
    \begin{envAnswer}[blankline=3]         $     3.5        $ \end{envAnswer}
    \begin{envAnswer}[blankline=0]         $   -15.25       $ \end{envAnswer}


\item $x^2-6x -10$ is to be written in the form $(x-p)^2+q$. Find the value of $p$ and the value of $q$.
    \begin{envAnswer}[blankline=3]         $     3.5        $ \end{envAnswer}
    \begin{envAnswer}[blankline=0]         $   -15.25       $ \end{envAnswer}


\item $2x^2+5x -10$ is to be written in the form $2(x+p)^2+q$. Find the value of $p$ and the value of $q$.
    \begin{envAnswer}[blankline=3]         $     3.5        $ \end{envAnswer}
    \begin{envAnswer}[blankline=0]         $   -15.25       $ \end{envAnswer}


\item The equation $5-\frac{3}{x^2}=-5x-2$ can be written in the form $x^3+ax^2+b=0$. Find the value of $a$ and the value of $b$.
    \begin{envAnswer}[blankline=3]         $     3.5        $ \end{envAnswer}
    \begin{envAnswer}[blankline=0]         $   -15.25       $ \end{envAnswer}




\end{enumerate}



\begin{bxTipTable}[colbacktitle=green]{Skill: Rearranging Formulae Check2  abcde98}{X|Y|Y}
\hline
Algebra & One & Two \\\hline
Algebra & $\frac{a}{\frac{a^2}{aaa^2 + b^2  }} $ \newline aaaa \newline bbbb  aaa  & Two \\[10pt]\hline
Algebra & \begin{equation*} \frac{a}{\frac{a^2}{aaa^2 + b^2  }}  \end{equation*}  & Two \\[10pt]\hline
Algebra & $\frac{a}{\frac{a^2}{aaa^2 + b^2  }} $& Two \\[10pt]\hline
Algebra & $\frac{a}{\frac{a^2}{aaa^2 + b^2  }} $& Two \\\hline
Algebra & $\frac{a}{\frac{a^2}{aaa^2 + b^2  }} $& Two \\\hline
Algebra & a $\frac{a}{\frac{a^2}{aaa^2 + b^2  }}  $  b  c & Two \\\hline
Algebra & $\frac{a}{\frac{a^2}{aaa^2 + b^2  }} $& Two \\\hline
Algebra & One & Two \\\hline
Algebra & One & Two 
\end{bxTipTable}


\begin{skilltable}[X|Y|Y]{Algebra  check1}
Algebra & One & Two \\\hline
Algebra & $\frac{a}{\frac{a^2}{aaa^2 + b^2  }} $ \newline aaaa \newline bbbb  aaa  & Two \\[10pt]\hline
Algebra & \begin{equation*} \frac{a}{\frac{a^2}{aaa^2 + b^2  }}  \end{equation*}  & Two \\[10pt]\hline
Algebra & $\frac{a}{\frac{a^2}{aaa^2 + b^2  }} $& Two \\[10pt]\hline
Algebra & $\frac{a}{\frac{a^2}{aaa^2 + b^2  }} $& Two \\\hline
Algebra & $\frac{a}{\frac{a^2}{aaa^2 + b^2  }} $& Two \\\hline
Algebra & a $\frac{a}{\frac{a^2}{aaa^2 + b^2  }}  $  b  c & Two \\\hline
Algebra & $\frac{a}{\frac{a^2}{aaa^2 + b^2  }} $& Two \\\hline
Algebra & One & Two \\\hline
Algebra & One & Two 
\end{skilltable}


\begin{skilltable}[X|Y|Y|Y|Y|Y]{My table}
Group & One & Two & Three & Four & Sum\\\hline\hline
Red & 1000.00 & 2000.00 & 3000.00 & 4000.00 & 10000.00\\\hline
Green & 2000.00 & 3000.00 & 4000.00 & 5000.00 & 14000.00\\\hline
Blue & 3000.00 & 4000.00 & 5000.00 & 6000.00 & 18000.00\\\hline\hline
Sum & 6000.00 & 9000.00 & 12000.00 & 15000.00 & 42000.00
\end{skilltable}


\begin{bxHighlight}{}    
a12345678911abc
\end{bxHighlight}

\begin{bxHighlight}[sharp corners]    
12345678911a
\end{bxHighlight}

\begin{bxHighlight}[sharp corners]{}    
12345678911b
\end{bxHighlight}


\begin{bxHighlight}[sharp corners]{sharp corners}    
12345678911c
\end{bxHighlight}


\begin{bxHighlight}{}    
12345678911
\end{bxHighlight}


\begin{bxExample}{}    
    Example: Solve the inequality $ \lvert x \rvert \geqslant 2x-10. $


\tcbline 



\begin{warningEnv}
     $ \lvert x \rvert \geqslant 2x-10 $ \\
     $ (x)^2 \geqslant (2x-10)^2 \quad \quad \quad $ \dots \dots Incorrect \\
     \dots \\
     \dots     
\end{warningEnv}

\tcbline


Note:
\begin{itemize}
       \item[] Above is one common error in solving absolute value inequlities as 
                \[    \lvert a \rvert \geqslant b       \centernot\implies     a^2 \geqslant b^2  \quad  \quad \text{if b is negative.} \]    
    
      \item[]  For example, $x=0$ is a valid value in the above inequality but
                \[     \lvert 0 \rvert \geqslant -10       \centernot\implies     0^2 \geqslant (-10)^2 \] 
\end{itemize}






    
\end{bxExample}
















% % \begin{warningEnv}
    % % $aaa x = x +1 $ Warning environment.\\
    % % Lorem \textcolor{blue}{000000000000000000} ipsum dolor sit amet\\
    % % Earum odit quia maiores nisi illum reiciendis aspernatur.
% % \end{warningEnv}


% % % \clearpage


% % % %--------------------------------------------------------------------------------------------------------------------------------------------------
% % % % Description:
% % % %     centernot package
% % % %
% % % % URL:
% % % %     \Rightarrow vs. \implies, and “does not imply” symbol
% % % %
% % % %     https://tex.stackexchange.com/questions/47063/rightarrow-vs-implies-and-does-not-imply-symbol
% % % %
% % % % Date: 2020.05.28
% % % %
% % % %-------------------------------------------------------------------------------------------------------------------------------------------------




% % % %\end{enumerate}



% % % % %Comment this line or Delete any commented line bellow    
% % % % \banBox{Ban box.}
 % % % % A
% % % % \begin{banEnv}
    % % % % Ban environment.\\
    % % % % Lorem ipsum dolor sit amet.\\
    % % % % Earum odit quia maiores nisi illum reiciendis aspernatur.
    
    
    
    
% % % % \end{banEnv}
 % % % % B
% % % % \warningBox{Warning box.}
 % % % % C
 % % % % D





















% % % % 111111111111111111111111111111\\



% % % % Question 8: \\

% % % % b)\\
% % % % The rate of change of h means: $ \frac{dh}{dt}     $\\. 

% % % % \begin{align*}
 % % % % \frac{dV}{dt}  &= \frac{dV}{dh} \times \frac{dh}{dt}  \\
  % % % % 24 &= 40\sqrt{3} \times 2h \times \frac{dh}{dt}     \\ 
  % % % % 24 &= 40\sqrt{3} \times 2 \times 12 \times \frac{dh}{dt}
% % % % \end{align*}


% % % % \clearpage




% % % % \awesomebox{5pt}{\faCertificate}{magenta}{Lorem ipsum…}  \\

% % % % \awesomebox{5pt}{\faFileExcel}{yellow}{ }


% % % % \begin{awesomeblock}[red]{5pt}{\faFileExcel[regular]}{yellow}
  % % % % $ (2x-1)^2 \geqslant (x+5)^2 $ \\
  % % % % aaaaaaaaaaaa  \\
  % % % % bbb \\
  % % % % ccc
% % % % \end{awesomeblock}





% % % % \awesomebox{5pt}{\faCertificate}{magenta}{Lorem ipsum…}

% % % % \awesomebox{5pt}{\faFileExcel}{yellow}{ }


% % % % \begin{awesomeblock}[red]{5pt}{\faFileExcel[regular]}{yellow}  
  
  % % % % aaaaaaaaaaaa \\
  % % % % bbbb  \\
% % % % \end{awesomeblock}




\begin{bxExample}{Factorising}


$ a^2 + b^2 = c^2 $ \\



$ a^2 + b^2 = c^2 $ \\

$ a^2 + b^2 = c^2 $ \\

\tcblower

$ a^2 + b^2 = c^2 $ \\


$ a^2 + b^2 = c^2 $ \\

$ a^2 + b^2 = c^2 $

\end{bxExample}



\begin{bxIntro}[Factorising]

KeyFind\\

$ a^2 + b^2 = c^2 $ \\



$ a^2 + b^2 = c^2 $ \\

$ a^2 + b^2 = c^2 $ \\

\tcblower

$ a^2 + b^2 = c^2 $ \\


$ a^2 + b^2 = c^2 $ \\

$ a^2 + b^2 = c^2 $

\end{bxIntro}

\begin{table}[ht]
    \centering
    \sffamily
    \begin{tabular}{| r | l | r |} \hline
        \rowcolor{blue!15} & \tblhead{Distribution} & \tblhead{Hits} \\  \hline 
            1 & Mint & 2364                                          \\  \hline  
            2 & Ubuntu & 1838                                        \\  \hline 
            3 & Debian & 1582                                        \\  \hline 
            4 & openSUSE & 1334                                      \\  \hline 
            4 & openSUSE & 1334                                      \\  \hline 
            5 & Fedora & 1262 \\
            6 & Mageia & 1219 \\
            7 & CentOS & 1171 \\
            8 & Arch & 1040 \\
            9 & elementary & 899 \\
            10 & Zorin & 851 \\
        \end{tabular}
\end{table}

\clearpage





\begin{tcbraster}[raster equal height,enhanced,
watermark text=\tcbsegmentstate]
\begin{tcolorbox}Upper part\end{tcolorbox}
\begin{tcolorbox}Upper part\tcblower Lower part\end{tcolorbox}
\end{tcbraster}


\clearpage



\noindent 
Every line is inside an equal height group:
\begin{tcbraster}[raster equal height,
                  raster column skip=-1pt,
                  title=Box \thetcbrasternum,
                  enhanced,
                  size=small,
                  styexample]
\begin{tcolorbox}First 1111 line\\second line\\


\begin{warningEnv}
     $ \lvert x \rvert \geqslant 2x-10 $ \\
     $ (x)^2 \geqslant (2x-10)^2 \quad \quad \quad $ \dots \dots Incorrect \\
     \dots \\
     \dots     
\end{warningEnv}



Note:
\begin{itemize}
       \item[] Above is one common error in solving absolute value inequlities as 
                \[    \lvert a \rvert \geqslant b       \centernot\implies     a^2 \geqslant b^2  \quad  \quad \text{if b is negative.} \]    
    
      \item[]  For example, $x=0$ is a valid value in the above inequality but
                \[     \lvert 0 \rvert \geqslant -10       \centernot\implies     0^2 \geqslant (-10)^2 \] 
\end{itemize}





The height of this box rules.\end{tcolorbox}
\begin{tcolorbox}[use height from group]Test\end{tcolorbox}
\begin{tcolorbox}
First line\\second line\end{tcolorbox}
\begin{tcolorbox}The height of this box rules.\end{tcolorbox}
\end{tcbraster}






\begin{tcbraster}[raster equal height,
enhanced,
title=tolorbox ABC  \thetcbrasternum]
\begin{tcolorbox}    
    Example: Solve the inequality $ \lvert x \rvert \geqslant 2x-10. $

\begin{warningEnv}
     $ \lvert x \rvert \geqslant 2x-10 $ \\
     $ (x)^2 \geqslant (2x-10)^2 \quad \quad \quad $ \dots \dots Incorrect \\
     \dots \\
     \dots     
\end{warningEnv}

 


Note:
\begin{itemize}
       \item[] Above is one common error in solving absolute value inequlities as 
                \[    \lvert a \rvert \geqslant b       \centernot\implies     a^2 \geqslant b^2  \quad  \quad \text{if b is negative.} \]    
    
      \item[]  For example, $x=0$ is a valid value in the above inequality but
                \[     \lvert 0 \rvert \geqslant -10       \centernot\implies     0^2 \geqslant (-10)^2 \] 
\end{itemize}
\end{tcolorbox}
\begin{tcolorbox}    
 ddddd\\
 
 ddd
 
    Example 1111111  : Solve
    \\
    the inequality \\
    $ \lvert x \rvert \geqslant 2x-10. $



$\frac{1}{1+3}$
\end{tcolorbox}
\begin{tcolorbox}
  ccccc\\
  ddd
  
\end{tcolorbox}
\begin{tcolorbox}aaa \end{tcolorbox}
\end{tcbraster}


\clearpage


EF Mark001  11111111111very line is inside an equal height group:
\begin{tcbraster}[raster columns=2, raster equal height=rows,
title=Box \thetcbrasternum,
enhanced,size=small]
\begin{tcolorbox}[colback=white,    before skip=2mm,
    after skip=2mm,
    colback=bkcolor,
	coltitle=black,
    colframe=black!50,
    boxrule=0.2mm,
    attach boxed title to top left={xshift=1cm,yshift*=1mm-\tcboxedtitleheight},
    varwidth boxed title*=-3cm,
    boxed title style={frame code={
                        \path[fill=tcbcolback!30!effbkcolor,
						      draw=blue!50]
                               ([yshift=-1mm,xshift=-1mm]frame.north west)
                               arc[start angle=0,end angle=180,radius=1mm]
                               ([yshift=-1mm,xshift=1mm]frame.north east)
                               arc[start angle=180,end angle=0,radius=1mm];
                        \path[left color=tcbcolback!60!effbkcolor,
						      right color=tcbcolback!60!effbkcolor,
                              middle color=tcbcolback!80!effbkcolor,
							  draw=black!50]
                               ([xshift=-2mm]frame.north west) -- ([xshift=2mm]frame.north east)
                               [rounded corners=1mm]-- ([xshift=1mm,yshift=-1mm]frame.north east)
                               -- (frame.south east) -- (frame.south west)
                               -- ([xshift=-1mm,yshift=-1mm]frame.north west)
                               [sharp corners]-- cycle;
                      },interior engine=empty,
    },
    fonttitle=\bfseries
    ]

$x^2$   abc def First line\\second line\\
The height of this box rules.\end{tcolorbox}
\begin{tcolorbox}[use height from group]Test\end{tcolorbox}
\begin{tcolorbox}
First line\\second line\end{tcolorbox}
\begin{tcolorbox}The height of this box rules.\end{tcolorbox}
\begin{bxTip}23The height of this box rules.\end{bxTip}
\begin{bxTip}12 The height of this box rules.\end{bxTip}
\end{tcbraster}






\clearpage

E Mark345  11111111111very line is inside an equal height group:
\begin{tcbraster}[raster equal height,
title=Box \thetcbrasternum,
enhanced,size=small]
\begin{tcolorbox}[colback=white,    before skip=2mm,
    after skip=2mm,
    colback=bkcolor,
	coltitle=black,
    colframe=black!50,
    boxrule=0.2mm,
    attach boxed title to top left={xshift=1cm,yshift*=1mm-\tcboxedtitleheight},
    varwidth boxed title*=-3cm,
    boxed title style={frame code={
                        \path[fill=tcbcolback!30!effbkcolor,
						      draw=blue!50]
                               ([yshift=-1mm,xshift=-1mm]frame.north west)
                               arc[start angle=0,end angle=180,radius=1mm]
                               ([yshift=-1mm,xshift=1mm]frame.north east)
                               arc[start angle=180,end angle=0,radius=1mm];
                        \path[left color=tcbcolback!60!effbkcolor,
						      right color=tcbcolback!60!effbkcolor,
                              middle color=tcbcolback!80!effbkcolor,
							  draw=black!50]
                               ([xshift=-2mm]frame.north west) -- ([xshift=2mm]frame.north east)
                               [rounded corners=1mm]-- ([xshift=1mm,yshift=-1mm]frame.north east)
                               -- (frame.south east) -- (frame.south west)
                               -- ([xshift=-1mm,yshift=-1mm]frame.north west)
                               [sharp corners]-- cycle;
                      },interior engine=empty,
    },
    fonttitle=\bfseries
    ]

$x^2$   abc def First line\\second line\\
The height of this box rules.\end{tcolorbox}
\begin{tcolorbox}[use height from group]Test\end{tcolorbox}
\begin{tcolorbox}
First line\\second line\end{tcolorbox}
\begin{tcolorbox}The height of this box rules.\end{tcolorbox}
\end{tcbraster}




E 11111111111 Mark123 very line is inside an equal height group:
\begin{tcbraster}[raster columns=2,raster equal height=rows,  raster valign=top,
title=Box \thetcbrasternum,
enhanced,size=small]
\begin{bxExample}[]First line\\second line\\
The height of this box rules.

\end{bxExample}
\begin{bxExample}[use height from group]Test
\end{bxExample}
\begin{bxExample}[]First line\\second line\\
The height of this box rules.

\end{bxExample}
\begin{bxExample}[use height from group, colback=white]Test 22222
\end{bxExample}
\begin{tcolorbox}[colback=white]
First line\\second line\end{tcolorbox}

\begin{tcolorbox}[colback=white]The height of this box rules.\end{tcolorbox}
\end{tcbraster}



\clearpage

Mark 988\\

\begin{tcbraster}[standard,      raster equal height=rows,raster columns=2,
colback=LightGreen,colframe=DarkGreen,colbacktitle=LimeGreen!75!DarkGreen,
left=1mm,right=1mm,top=1mm,bottom=1mm,middle=1mm]

\begin{tcolorbox}
    Example: Solve the inequality $ \lvert x \rvert \geqslant 2x-10. $


\tcbline 



\begin{warningEnv}
     $ \lvert x \rvert \geqslant 2x-10 $ \\
     $ (x)^2 \geqslant (2x-10)^2 \quad \quad \quad $ \dots \dots Incorrect \\
     \dots \\
     \dots     
\end{warningEnv}

\tcbline


Note:
\begin{itemize}
       \item[] Above is one common error in solving absolute value inequlities as 
                \[    \lvert a \rvert \geqslant b       \centernot\implies     a^2 \geqslant b^2  \quad  \quad \text{if b is negative.} \]    
    
      \item[]  For example, $x=0$ is a valid value in the above inequality but
                \[     \lvert 0 \rvert \geqslant -10       \centernot\implies     0^2 \geqslant (-10)^2 \] 
\end{itemize}






    
\end{tcolorbox}
\begin{tcolorbox}
    Example: Solve the inequality $ \lvert x \rvert \geqslant 2x-10. $


\tcbline 



\begin{warningEnv}
     $ \lvert x \rvert \geqslant 2x-10 $ \\
     $ (x)^2 \geqslant (2x-10)^2 \quad \quad \quad $ \dots \dots Incorrect \\
     \dots \\
     \dots     
\end{warningEnv}

\tcbline


Note:
\begin{itemize}
       \item[] Above is one common error in solving absolute value inequlities as 
                \[    \lvert a \rvert \geqslant b       \centernot\implies     a^2 \geqslant b^2  \quad  \quad \text{if b is negative.} \]    
    
      \item[]  For example, $x=0$ is a valid value in the above inequality but
                \[     \lvert 0 \rvert \geqslant -10       \centernot\implies     0^2 \geqslant (-10)^2 \] 
\end{itemize}
\end{tcolorbox}

\begin{tcolorbox}
This is my content.
\end{tcolorbox}
\begin{tcolorbox}
This is my content.
\tcblower
More content.
\end{tcolorbox}
\begin{tcolorbox}[adjusted title=My title]
This is my content.
\end{tcolorbox}
\begin{tcolorbox}[adjusted title=My title]
This is my content.
\tcblower
More content.
\end{tcolorbox}
\end{tcbraster}

% \begin{YetTheorem}{Mittelwertsatz f\"{u}r $n$ Variable}{mittelwertsatz_n3}%
% Es sei $n\in\mathbb{N}$, $D\subseteq\mathbb{R}^n$ eine offene Menge und
% $f\in C^{1}(D,\mathbb{R})$. Dann gibt es auf jeder Strecke
% $[x_0,x]\subset D$ einen Punkt $\xi\in[x_0,x]$, so dass gilt
% \begin{equation*}
% f(x)-f(x_0) = \operatorname{grad} f(\xi)^{\top}(x-x_0)
% \end{equation*}
% \end{YetTheorem}



% \begin{YetAnotherTheorem} %[Mittelwertsatz f\"{u}r $n$ Variable]
             % %{mittelwertsatz_n3}%
% Es sei $n\in\mathbb{N}$, $D\subseteq\mathbb{R}^n$ eine offene Menge und
% $f\in C^{1}(D,\mathbb{R})$. Dann gibt es auf jeder Strecke
% $[x_0,x]\subset D$ einen Punkt $\xi\in[x_0,x]$, so dass gilt
% \begin{equation*}
% f(x)-f(x_0) = \operatorname{grad} f(\xi)^{\top}(x-x_0)
% \end{equation*}
% \end{YetAnotherTheorem}
