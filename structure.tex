%%%%%%%%%%%%%%%%%%%%%%%%%%%%%%%%%%%%%%%%%%%%%%%%%%%%%%%%%%%%%%%%%%%%%%%%%%%%%%%%%%%%%%%%%%%%%%%%%%%%%%%%%%%%%
% Structural Definitions File
% Version 2.0 (9/2/15)
%
% Original author:
% Mathias Legrand (legrand.mathias@gmail.com) with modifications by:
% Vel (vel@latextemplates.com)
% 
% This file has been downloaded from:
% http://www.LaTeXTemplates.com
%
% License:
% CC BY-NC-SA 3.0 (http://creativecommons.org/licenses/by-nc-sa/3.0/)
%
%%%%%%%%%%%%%%%%%%%%%%%%%%%%%%%%%%%%%%%%%%%%%%%%%%%%%%%%%%%%%%%%%%%%%%%%%%%%%%%%%%%%%%%%%%%%%%%%%%%%%%%%%%%%%




%----------------------------------------------------------------------------------------
%    
%----------------------------------------------------------------------------------------
\providecommand{\varFolderA}{}
\providecommand{\varExamCode}{}
\providecommand{\varExamVar}{}

\providecommand{\txtTitleA}{}
\providecommand{\txtTitleB}{}
\providecommand{\txtTitleC}{}
\providecommand{\txtVersion}{}
\providecommand{\txtFoot}{}
\providecommand{\txtEmail}{}



\providecommand{\iPageStart}{1}

%----------------------------------------------------------------------------------------
%    VARIOUS REQUIRED PACKAGES AND CONFIGURATIONS
%----------------------------------------------------------------------------------------
\usepackage{dashundergaps}
\usepackage{anyfontsize}
\usepackage{qrcode}
\usepackage{pgfkeys}
\usepackage{xparse}
\usepackage{csvsimple}
\usepackage{gensymb}
\usepackage{ifthen}

\usepackage{colortbl}
\usepackage{xcolor} % Required for specifying colors by name
\usepackage{pgf,pgfplots}
\usepackage{tikz}         % Required for drawing custom shapes
\usepackage{tkz-fct}
\usepackage{tkz-euclide}

\pgfplotsset{compat=1.15}
\usepackage{mathrsfs}
\usetikzlibrary{3d,angles,arrows,bending,calc,decorations.softclip,fillbetween,matrix,positioning,quotes,shapes}

\usepackage{pifont,manfnt,bbding}

\usepackage[top=2cm,bottom=2cm,left=2cm,right=2cm,headsep=10pt,a4paper]{geometry} % Page margins

\usepackage{graphicx} % Required for including pictures
\graphicspath{{Pictures/}} % Specifies the directory where pictures are stored

\usepackage{lipsum} % Inserts dummy text

\usepackage[explicit]{titlesec}
\usepackage{varwidth}

\definecolor{ocre}{RGB}{243,102,25} % Define the orange color used for highlighting throughout the book
\definecolor{structurecolor}{RGB}{0,120,2}% From Elegantbook - Green


\usepackage[english]{babel} % English language/hyphenation

\usepackage{enumitem} % Customize lists
%\setlist{nolistsep} % Reduce spacing between bullet points and numbered lists

\usepackage{booktabs} % Required for nicer horizontal rules in tables


\usepackage{xstring}
\usepackage{multido}
\usepackage{array}
\usepackage{tabularx}
\usepackage{witharrows}
\usepackage{paracol}




%----------------------------------------------------------------------------------------
%  Function:
%      Straight "upright" integration symbol
%
%  Webpage:
%      https://tex.stackexchange.com/questions/503527/how-to-write-upright-integrals-with-automatic-sizing
%      https://tex.stackexchange.com/questions/425098/which-opentype-math-fonts-are-available
%
%  Date:
%      2023.08.22
%
%----------------------------------------------------------------------------------------


%\usepackage{unicode-math}
%\usepackage[upint]{stix}

%\setmathfont[StylisticSet=8]{STIX Two Math}

%\usepackage[integrals]{wasysym}
\usepackage[nolimits]{cmupint}


%----------------------------------------------------------------------------------------
%    FONTS
%----------------------------------------------------------------------------------------

\usepackage{avant} % Use the Avantgarde font for headings
%\usepackage{times} % Use the Times font for headings

\usepackage{mathptmx} % Use the Adobe Times Roman as the default text font 
                      % together with math symbols from the Symbol, 
					  % Chancery and Computer Modern fonts


\usepackage{microtype} % Slightly tweak font spacing for aesthetics
\usepackage[utf8]{inputenc} % Required for including letters with accents
\usepackage[T1]{fontenc} % Use 8-bit encoding that has 256 glyphs

%----------------------------------------------------------------------------------------
%    BIBLIOGRAPHY AND INDEX
%----------------------------------------------------------------------------------------

\usepackage[style=numeric,citestyle=numeric,sorting=nyt
                         ,sortcites=true,autopunct=true
						 ,babel=hyphen,hyperref=true
						 ,abbreviate=false,backref=true,backend=biber]{biblatex}
\addbibresource{bibliography.bib} % BibTeX bibliography file
\defbibheading{bibempty}{}

\usepackage{calc} % For simpler calculation - used for spacing the index letter headings correctly
\usepackage{makeidx} % Required to make an index
\makeindex % Tells LaTeX to create the files required for indexing

%----------------------------------------------------------------------------------------
%    MAIN TABLE OF CONTENTS
%----------------------------------------------------------------------------------------

\usepackage{titletoc} % Required for manipulating the table of contents

\contentsmargin{0cm} % Removes the default margin

% Part text styling
\titlecontents{part}[0cm]
{\addvspace{20pt}\centering\large\bfseries}
{}
{}
{}

% Chapter text styling
\titlecontents{chapter}[1.25cm] % Indentation
{\addvspace{12pt}\large\sffamily\bfseries} % Spacing and font options for chapters
{\color{ocre!60}\contentslabel[\Large\thecontentslabel]{1.25cm}\color{ocre}} % Chapter number
{\color{ocre}}  
{\color{ocre!60}\normalsize\;\titlerule*[.5pc]{.}\;\thecontentspage} % Page number

% Section text styling
\titlecontents{section}[1.25cm] % Indentation
{\addvspace{3pt}\sffamily\bfseries} % Spacing and font options for sections
{\contentslabel[\thecontentslabel]{1.25cm}} % Section number
{}
{\hfill\color{black}\thecontentspage} % Page number
[]

% Subsection text styling
\titlecontents{subsection}[1.25cm] % Indentation
{\addvspace{1pt}\sffamily\small} % Spacing and font options for subsections
{\contentslabel[\thecontentslabel]{1.25cm}} % Subsection number
{}
{\ \titlerule*[.5pc]{.}\;\thecontentspage} % Page number
[]

% List of figures
\titlecontents{figure}[0em]
{\addvspace{-5pt}\sffamily}
{\thecontentslabel\hspace*{1em}}
{}
{\ \titlerule*[.5pc]{.}\;\thecontentspage}
[]

% List of tables
\titlecontents{table}[0em]
{\addvspace{-5pt}\sffamily}
{\thecontentslabel\hspace*{1em}}
{}
{\ \titlerule*[.5pc]{.}\;\thecontentspage}
[]

%----------------------------------------------------------------------------------------
%    MINI TABLE OF CONTENTS IN PART HEADS
%----------------------------------------------------------------------------------------

% Chapter text styling
\titlecontents{lchapter}[0em] % Indenting
{\addvspace{15pt}\large\sffamily\bfseries} % Spacing and font options for chapters
{\color{ocre}\contentslabel[\Large\thecontentslabel]{1.25cm}\color{ocre}} % Chapter number
{}  
{\color{ocre}\normalsize\sffamily\bfseries\;\titlerule*[.5pc]{.}\;\thecontentspage} % Page number

% Section text styling
\titlecontents{lsection}[0em] % Indenting
{\sffamily\small} % Spacing and font options for sections
{\contentslabel[\thecontentslabel]{1.25cm}} % Section number
{}
{}

% Subsection text styling
\titlecontents{lsubsection}[.5em] % Indentation
{\normalfont\footnotesize\sffamily} % Font settings
{}
{}
{}

%----------------------------------------------------------------------------------------
%    THEOREM STYLES
%----------------------------------------------------------------------------------------

\usepackage{amsmath,amsfonts,amssymb,amsthm} % For math equations, theorems, symbols, etc
\usepackage{mathtools}
\usepackage{latexsym}
\usepackage{bbding}

%-------------------------------------------------------------------------------------------------------------------------
% https://en.wikibooks.org/wiki/LaTeX/Special_Characters#Degree_symbol_for_temperature_and_math
%
%-------------------------------------------------------------------------------------------------------------------------
\usepackage{siunitx}

\newcommand{\intoo}[2]{\mathopen{]}#1\,;#2\mathclose{[}}
\newcommand{\ud}{\mathop{\mathrm{{}d}}\mathopen{}}
\newcommand{\intff}[2]{\mathopen{[}#1\,;#2\mathclose{]}}
\newtheorem{notation}{Notation}[chapter]

% Boxed/framed environments
\newtheoremstyle{ocrenumbox}% % Theorem style name
{0pt}% Space above
{0pt}% Space below
{\normalfont}% % Body font
{}% Indent amount
{\small\bf\sffamily\color{ocre}}% % Theorem head font
{\;}% Punctuation after theorem head
{0.25em}% Space after theorem head
{\small\sffamily\color{ocre}\thmname{#1}\nobreakspace\thmnumber{\@ifnotempty{#1}{}\@upn{#2}}% Theorem text (e.g. Theorem 2.1)
\thmnote{\nobreakspace\the\thm@notefont\sffamily\bfseries\color{black}---\nobreakspace#3.}} % Optional theorem note
\renewcommand{\qedsymbol}{$\blacksquare$}% Optional qed square

\newtheoremstyle{blacknumex}% Theorem style name
{5pt}% Space above
{5pt}% Space below
{\normalfont}% Body font
{} % Indent amount
{\small\bf\sffamily}% Theorem head font
{\;}% Punctuation after theorem head
{0.25em}% Space after theorem head
{\small\sffamily{\tiny\ensuremath{\blacksquare}}\nobreakspace\thmname{#1}\nobreakspace\thmnumber{\@ifnotempty{#1}{}\@upn{#2}}% Theorem text (e.g. Theorem 2.1)
\thmnote{\nobreakspace\the\thm@notefont\sffamily\bfseries---\nobreakspace#3.}}% Optional theorem note

\newtheoremstyle{blacknumbox} % Theorem style name
{0pt}% Space above
{0pt}% Space below
{\normalfont}% Body font
{}% Indent amount
{\small\bf\sffamily}% Theorem head font
{\;}% Punctuation after theorem head
{0.25em}% Space after theorem head
{\small\sffamily\thmname{#1}\nobreakspace\thmnumber{\@ifnotempty{#1}{}\@upn{#2}}% Theorem text (e.g. Theorem 2.1)
\thmnote{\nobreakspace\the\thm@notefont\sffamily\bfseries---\nobreakspace#3.}}% Optional theorem note

% Non-boxed/non-framed environments
\newtheoremstyle{ocrenum}% % Theorem style name
{5pt}% Space above
{5pt}% Space below
{\normalfont}% % Body font
{}% Indent amount
{\small\bf\sffamily\color{ocre}}% % Theorem head font
{\;}% Punctuation after theorem head
{0.25em}% Space after theorem head
{\small\sffamily\color{ocre}\thmname{#1}\nobreakspace\thmnumber{\@ifnotempty{#1}{}\@upn{#2}}% Theorem text (e.g. Theorem 2.1)
\thmnote{\nobreakspace\the\thm@notefont\sffamily\bfseries\color{black}---\nobreakspace#3.}} % Optional theorem note
\renewcommand{\qedsymbol}{$\blacksquare$}% Optional qed square
\makeatother

% Defines the theorem text style for each type of theorem to one of the three styles above
\newcounter{dummy} 
\numberwithin{dummy}{section}
\theoremstyle{ocrenumbox}
\newtheorem{theoremeT}[dummy]{Theorem}
\newtheorem{problem}{Problem}[chapter]
\newtheorem{exerciseT}{Exercise}[chapter]
\theoremstyle{blacknumex}
\newtheorem{exampleT}{Example}[chapter]
\theoremstyle{blacknumbox}
\newtheorem{vocabulary}{Vocabulary}[chapter]
\newtheorem{definitionT}{Definition}[section]
\newtheorem{corollaryT}[dummy]{Corollary}
\theoremstyle{ocrenum}
\newtheorem{proposition}[dummy]{Proposition}

%----------------------------------------------------------------------------------------
%    DEFINITION OF COLORED BOXES
%----------------------------------------------------------------------------------------

\RequirePackage[framemethod=default]{mdframed} % Required for creating the theorem, definition, exercise and corollary boxes

% Theorem box
\newmdenv[skipabove=7pt,
skipbelow=7pt,
backgroundcolor=black!5,
linecolor=ocre,
innerleftmargin=5pt,
innerrightmargin=5pt,
innertopmargin=5pt,
leftmargin=0cm,
rightmargin=0cm,
innerbottommargin=5pt]{tBox}

% Exercise box      
\newmdenv[skipabove=7pt,
skipbelow=7pt,
rightline=false,
leftline=true,
topline=false,
bottomline=false,
backgroundcolor=ocre!10,
linecolor=ocre,
innerleftmargin=5pt,
innerrightmargin=5pt,
innertopmargin=5pt,
innerbottommargin=5pt,
leftmargin=0cm,
rightmargin=0cm,
linewidth=4pt]{eBox}    

% Definition box
\newmdenv[skipabove=7pt,
skipbelow=7pt,
rightline=false,
leftline=true,
topline=false,
bottomline=false,
linecolor=ocre,
innerleftmargin=5pt,
innerrightmargin=5pt,
innertopmargin=0pt,
leftmargin=0cm,
rightmargin=0cm,
linewidth=4pt,
innerbottommargin=0pt]{dBox}    

% Corollary box
\newmdenv[skipabove=7pt,
skipbelow=7pt,
rightline=false,
leftline=true,
topline=false,
bottomline=false,
linecolor=gray,
backgroundcolor=black!5,
innerleftmargin=5pt,
innerrightmargin=5pt,
innertopmargin=5pt,
leftmargin=0cm,
rightmargin=0cm,
linewidth=4pt,
innerbottommargin=5pt]{cBox}

% Creates an environment for each type of theorem and assigns it a theorem text style from the "Theorem Styles" section above and a colored box from above


%-------------------------------------------------------------------------------
% Note: Comment by Guo DeLiang
% Date: 15/02/2020
%
%-------------------------------------------------------------------------------
%% \newenvironment{theorem}{\begin{tBox}\begin{theoremeT}}{\end{theoremeT}\end{tBox}}

% \newtcbtheorem[auto counter,number within=section]{theo}%
  % {Theorem}{fonttitle=\bfseries\upshape, fontupper=\slshape,
     % arc=0mm, colback=blue!5!white,colframe=blue!75!black}{theorem}

%----------------------------------------------------------------------------
% Comment to avoid compile error
%
%%%%% \newenvironment{exercise}{\begin{eBox}\begin{exerciseT}}{\hfill{\color{ocre}\tiny\ensuremath{\blacksquare}}\end{exerciseT}\end{eBox}}                  


\newenvironment{definition}{\begin{dBox}\begin{definitionT}}{\end{definitionT}\end{dBox}}    
\newenvironment{example}{\begin{exampleT}}{\hfill{\tiny\ensuremath{\blacksquare}}\end{exampleT}}        
\newenvironment{corollary}{\begin{cBox}\begin{corollaryT}}{\end{corollaryT}\end{cBox}}    

%----------------------------------------------------------------------------------------
%    REMARK ENVIRONMENT
%----------------------------------------------------------------------------------------

\newenvironment{remark}{\par\vspace{10pt}\small % Vertical white space above the remark and smaller font size
\begin{list}{}{    
\leftmargin=35pt % Indentation on the left
\rightmargin=25pt}\item\ignorespaces % Indentation on the right
\makebox[-2.5pt]{\begin{tikzpicture}[overlay]
\node[draw=ocre!60,line width=1pt,circle,fill=ocre!25,font=\sffamily\bfseries,inner sep=2pt,outer sep=0pt] at (-15pt,0pt){\textcolor{ocre}{R}};\end{tikzpicture}} % Orange R in a circle
\advance\baselineskip -1pt}{\end{list}\vskip5pt} % Tighter line spacing and white space after remark

%----------------------------------------------------------------------------------------
%    SECTION NUMBERING IN THE MARGIN
%----------------------------------------------------------------------------------------

\makeatletter
\renewcommand{\@seccntformat}[1]{\llap{\textcolor{ocre}{\csname the#1\endcsname}\hspace{1em}}}                    
\renewcommand{\section}{\@startsection{section}{1}{\z@}
{-4ex \@plus -1ex \@minus -.4ex}
{1ex \@plus.2ex }
{\normalfont\large\sffamily\bfseries}}
\renewcommand{\subsection}{\@startsection {subsection}{2}{\z@}
{-3ex \@plus -0.1ex \@minus -.4ex}
{0.5ex \@plus.2ex }
{\normalfont\sffamily\bfseries}}
\renewcommand{\subsubsection}{\@startsection {subsubsection}{3}{\z@}
{-2ex \@plus -0.1ex \@minus -.2ex}
{.2ex \@plus.2ex }
{\normalfont\small\sffamily\bfseries}}                        
\renewcommand\paragraph{\@startsection{paragraph}{4}{\z@}
{-2ex \@plus-.2ex \@minus .2ex}
{.1ex}
{\normalfont\small\sffamily\bfseries}}

%----------------------------------------------------------------------------------------
%    PART HEADINGS
%----------------------------------------------------------------------------------------

% numbered part in the table of contents
\newcommand{\@mypartnumtocformat}[2]{%
\setlength\fboxsep{0pt}%
\noindent\colorbox{ocre!20}{\strut\parbox[c][.7cm]{\ecart}{\color{ocre!70}\Large\sffamily\bfseries\centering#1}}\hskip\esp\colorbox{ocre!40}{\strut\parbox[c][.7cm]{\linewidth-\ecart-\esp}{\Large\sffamily\centering#2}}}%
%%%%%%%%%%%%%%%%%%%%%%%%%%%%%%%%%%
% unnumbered part in the table of contents
\newcommand{\@myparttocformat}[1]{%
\setlength\fboxsep{0pt}%
\noindent\colorbox{ocre!40}{\strut\parbox[c][.7cm]{\linewidth}{\Large\sffamily\centering#1}}}%
%%%%%%%%%%%%%%%%%%%%%%%%%%%%%%%%%%
\newlength\esp
\setlength\esp{4pt}
\newlength\ecart
\setlength\ecart{1.2cm-\esp}
\newcommand{\thepartimage}{}%
\newcommand{\partimage}[1]{\renewcommand{\thepartimage}{#1}}%
\def\@part[#1]#2{%
\ifnum \c@secnumdepth >-2\relax%
\refstepcounter{part}%
\addcontentsline{toc}{part}{\texorpdfstring{\protect\@mypartnumtocformat{\thepart}{#1}}{\partname~\thepart\ ---\ #1}}
\else%
\addcontentsline{toc}{part}{\texorpdfstring{\protect\@myparttocformat{#1}}{#1}}%
\fi%
\startcontents%
\markboth{}{}%
{\thispagestyle{empty}%
\begin{tikzpicture}[remember picture,overlay]%
\node at (current page.north west){\begin{tikzpicture}[remember picture,overlay]%    
\fill[ocre!20](0cm,0cm) rectangle (\paperwidth,-\paperheight);
\node[anchor=north] at (4cm,-3.25cm){\color{ocre!40}\fontsize{220}{100}\sffamily\bfseries\thepart}; 
\node[anchor=south east] at (\paperwidth-1cm,-\paperheight+1cm){\parbox[t][][t]{8.5cm}{
\printcontents{l}{0}{\setcounter{tocdepth}{1}}%
}};
\node[anchor=north east] at (\paperwidth-1.5cm,-3.25cm){\parbox[t][][t]{15cm}{\strut\raggedleft\color{white}\fontsize{30}{30}\sffamily\bfseries#2}};
\end{tikzpicture}};
\end{tikzpicture}}%
\@endpart}
\def\@spart#1{%
\startcontents%
\phantomsection
{\thispagestyle{empty}%
\begin{tikzpicture}[remember picture,overlay]%
\node at (current page.north west){\begin{tikzpicture}[remember picture,overlay]%    
\fill[ocre!20](0cm,0cm) rectangle (\paperwidth,-\paperheight);
\node[anchor=north east] at (\paperwidth-1.5cm,-3.25cm){\parbox[t][][t]{15cm}{\strut\raggedleft\color{white}\fontsize{30}{30}\sffamily\bfseries#1}};
\end{tikzpicture}};
\end{tikzpicture}}
\addcontentsline{toc}{part}{\texorpdfstring{%
\setlength\fboxsep{0pt}%
\noindent\protect\colorbox{ocre!40}{\strut\protect\parbox[c][.7cm]{\linewidth}{\Large\sffamily\protect\centering #1\quad\mbox{}}}}{#1}}%
\@endpart}
\def\@endpart{\vfil\newpage
\if@twoside
\if@openright
\null
\thispagestyle{empty}%
\newpage
\fi
\fi
\if@tempswa
\twocolumn
\fi}

%----------------------------------------------------------------------------------------
%    CHAPTER HEADINGS
%----------------------------------------------------------------------------------------

% A switch to conditionally include a picture, implemented by  Christian Hupfer
\newif\ifusechapterimage
\usechapterimagetrue
\newcommand{\thechapterimage}{}%
\newcommand{\chapterimage}[1]{\ifusechapterimage\renewcommand{\thechapterimage}{#1}\fi}%
\newcommand{\autodot}{.}
\def\@makechapterhead#1{%
{\parindent \z@ \raggedright \normalfont
\ifnum \c@secnumdepth >\m@ne
\if@mainmatter
\begin{tikzpicture}[remember picture,overlay]
\node at (current page.north west)
{\begin{tikzpicture}[remember picture,overlay]
\node[anchor=north west,inner sep=0pt] at (0,0) {\ifusechapterimage\includegraphics[width=\paperwidth]{\thechapterimage}\fi};
\draw[anchor=west] (\Gm@lmargin,-9cm) node [line width=2pt,rounded corners=15pt,draw=ocre,fill=white,fill opacity=0.5,inner sep=15pt]{\strut\makebox[22cm]{}};
\draw[anchor=west] (\Gm@lmargin+.3cm,-9cm) node {\huge\sffamily\bfseries\color{black}\thechapter\autodot~#1\strut};
\end{tikzpicture}};
\end{tikzpicture}
\else
\begin{tikzpicture}[remember picture,overlay]
\node at (current page.north west)
{\begin{tikzpicture}[remember picture,overlay]
\node[anchor=north west,inner sep=0pt] at (0,0) {\ifusechapterimage\includegraphics[width=\paperwidth]{\thechapterimage}\fi};
\draw[anchor=west] (\Gm@lmargin,-9cm) node [line width=2pt,rounded corners=15pt,draw=ocre,fill=white,fill opacity=0.5,inner sep=15pt]{\strut\makebox[22cm]{}};
\draw[anchor=west] (\Gm@lmargin+.3cm,-9cm) node {\huge\sffamily\bfseries\color{black}#1\strut};
\end{tikzpicture}};
\end{tikzpicture}
\fi\fi\par\vspace*{270\p@}}}

%-------------------------------------------

\def\@makeschapterhead#1{%
\begin{tikzpicture}[remember picture,overlay]
\node at (current page.north west)
{\begin{tikzpicture}[remember picture,overlay]
\node[anchor=north west,inner sep=0pt] at (0,0) {\ifusechapterimage\includegraphics[width=\paperwidth]{\thechapterimage}\fi};
\draw[anchor=west] (\Gm@lmargin,-9cm) node [line width=2pt,rounded corners=15pt,draw=ocre,fill=white,fill opacity=0.5,inner sep=15pt]{\strut\makebox[22cm]{}};
\draw[anchor=west] (\Gm@lmargin+.3cm,-9cm) node {\huge\sffamily\bfseries\color{black}#1\strut};
\end{tikzpicture}};
\end{tikzpicture}
\par\vspace*{270\p@}}
\makeatother

%----------------------------------------------------------------------------------------
%    HYPERLINKS IN THE DOCUMENTS
%----------------------------------------------------------------------------------------

            
\usepackage[colorlinks=true,linkcolor=blue,urlcolor=black,bookmarksopen=true]{hyperref}
            
\usepackage{bookmark}
\bookmarksetup{
           open,
           % numbered,
           addtohook={%
                        \ifnum\bookmarkget{level}=0 % chapter
                             \bookmarksetup{bold}%
                        \fi

                        \ifnum\bookmarkget{level}=-1 % part
                             \bookmarksetup{color=ocre,bold}%
                        \fi
                      }
}




%----------------------------------------------------------------------------------------
%    Added by Guo De Liang Thomas
%----------------------------------------------------------------------------------------
\usepackage[SC5b]{ean13isbn}
\usepackage{todonotes}
\usepackage{multicol}
\usepackage{multiaudience}
\usepackage{etoolbox}
\usepackage{arydshln}


% \SetNewAudience{Tutor}
% \SetNewAudience{Parent}
% \SetNewAudience{Student}
% \SetNewAudience{Solution}
% \SetNewAudience{MarkScheme}


\usepackage{versions}
\usepackage[final]{pdfpages}
\usepackage{tasks}
\usepackage[makeroom]{cancel}

\usepackage{CJKutf8}               % Note: For Chinese Characters
%\usepackage{ctex}

\usepackage{tabto}

\let\exercise\pkexercise


%---------------------------------------------------------------------------------------------
% Description:
%     Set the type of final output
%
% Sample:
%     biBook       - Ebook
%
%     biPastpaper  - Due to copyright issue, need to set header, trailer
%                    
%                    
%
%     biTopical    - 
%     
%
% Date: 2019.08.15
%
%     bi           - Boolean Indicator
%
%--------------------------------------------------------------------------------------------

\newtoggle{biPublish}              % Published Book
                                   %     - All materials should be legal and copyright should 
                                   %       be verified.                                   

\togglefalse{biPublish}


\newtoggle{biCompact}              % Compact Format, no blank space or line
\togglefalse{biCompact}

\newtoggle{biTutorCN}              % Tutition in Chinese
\toggletrue{biTutorCN}

\newtoggle{biPrintOnly}            % bi Boolean Indicator - Print Only
\newtoggle{biNoCoverPage}          % bi Boolean Indicator - No Print Cover Page

\newtoggle{biStudent}              % bi Boolean Indicator - Student

\newtoggle{biPastPaper}            % bi Boolean Indicator - Past Paper
\newtoggle{biTopicTest}            % bi Boolean Indicator - Topic Test
\newtoggle{biWorksheet}            % bi Boolean Indicator - Worksheet
\newtoggle{biExcercise}            % bi Boolean Indicator - Excercise


\newtoggle{biSolution}             % bi Boolean Indicator - Solution
\newtoggle{biMarkScheme}           % bi Boolean Indicator - Mark Scheme



\newtoggle{biTutor}                % bi Boolean Indicator - Tutor


\newtoggle{biChinese}              % bi Boolean Indicator - Include Chinese.
\toggletrue{biChinese}     


%-----------------------------------------------------------------------------------------------------------------------
% Option Setup:
%     1. bTextbook
%     2. bRevision
%     3. bWorkbook
%     4. bTest
%     5. bPastPaper
%
%     6. bQuestion
%     7. bAnswer
%     8. bWorkedSolL           - Indicator Worked Solution Long
%     9. bWorkedSolS           - Indicator Worked Solution Short

%    10. bCompact              - Indicator if printed blank space
%------------------------------------------------------------------------------------------------------------------------

\newtoggle{bRef}

\newtoggle{bTextbook}
\newtoggle{bRevision}
\newtoggle{bWorkbook}
\newtoggle{bTest}
\newtoggle{bPastPaper}

\newtoggle{bQuestion}
\newtoggle{bSolution}
\newtoggle{bAnswer}
\newtoggle{bWorkedSolL}
\newtoggle{bWorkedSolS}

\newtoggle{bDottedLine}
\newtoggle{bCompact}

\newtoggle{biMacleans}             % bi Boolean Indicator
\newtoggle{biStKent}
\newtoggle{biACGStrathAllan}

\newcounter{nOddEven}

\newcounter{nExamYear}

\newcounter{nLevelStart}
\setcounter{nLevelStart}{1}    %------ Level

\newcounter{nLevelEnd}
\setcounter{nLevelEnd}{1800}   %------ Level


\newcounter{nWorkBook}



%-----------------------------------------------------------------------------------------------------
% Low Level Parameter Setup
%
%
%
%
%-----------------------------------------------------------------------------------------------------

\toggletrue{bRef}

\togglefalse{bTextbook}
\toggletrue{bRevision}
\togglefalse{bWorkbook}
\toggletrue{bTest}
\togglefalse{bPastPaper}


\togglefalse{biPrintOnly}
\toggletrue{biNoCoverPage}

\iftoggle{biNoCoverPage}{ \renewcommand{\iPageStart}{2} }
                        { \renewcommand{\iPageStart}{1} }

\toggletrue{biTutor}
\toggletrue{biStudent}
%\toggletrue{biBook}
%\toggletrue{biSolution}

\togglefalse{biMacleans}
%\toggletrue{biStKent}
%\toggletrue{biACGStrathAllan} 

\toggletrue{bDottedLine}
\togglefalse{bCompact}


\toggletrue{bQuestion}
\togglefalse{bSolution}
\togglefalse{bAnswer}
\togglefalse{bWorkedSolL}
\togglefalse{bWorkedSolS}

\setcounter{nExamYear}{00}     %------ Exam Year 2017


\setcounter{nWorkBook}{2}      %------ 0 Workbook A - Year 2017, 2015 
                               %------ 1 Workbook B - Year 2016, 2014
                               %------ 2 Workbook   - Year 2017, 2016, 2015, 2014....
                             



%-----------------------------------------------------------------------------------------------------
% High Level Parameter Setup
%
%
%
%
%-----------------------------------------------------------------------------------------------------
\iftoggle{bAnswer}{ \toggletrue{bSolution}}{}

%-----------------------------------------------------------------------------------------------------------------------------------------
%
% Function:
%          The setting for printing
%              1. Set the alternate row color to white
%
%
%     
%
% Date: 2020.08.06
%
%-----------------------------------------------------------------------------------------------------------------------------------------

\iftoggle{biPrintOnly}{
    % Alternate Row Color to white
    \colorlet{bkcolor}{white}
	\colorlet{effbkcolor}{white}
}
{
    \colorlet{bkcolor}{gray!15}
	\colorlet{effbkcolor}{white}
}


%-----------------------------------------------------------------------------------------------------------------------------------------
%
% URL:
%     Use a class as a package? (or parts of the class)
%     https://tex.stackexchange.com/questions/272239/use-a-class-as-a-package-or-parts-of-the-class/272250#272250
%
%
% Function:
%     
%     
%
%     examimport.sty
%     
%              \ProvidesPackage{examimport}
%
%              \RequirePackage{xparse}
%
%              \RenewDocumentCommand{\ProvidesClass}{mO{}}{%
%                   }
%
%              \RenewDocumentCommand{\LoadClass}{O{}mO{}}{%
%                   }
%
%              \input{examcopy.cls}
%
%\endinput
%
% Date: 2020.05.05
%
%-----------------------------------------------------------------------------------------------------------------------------------------







\usepackage{booktabs} % Required for nicer horizontal rules in tables
\usepackage{tcolorbox}
\tcbuselibrary{breakable, skins, listings,raster,poster, most}


%\usepackage{fontawesome}
\usepackage{fontawesome5}
\usepackage{awesomebox}
\usepackage{intcalc}
\setcounter{tocdepth}{5}
\usepackage{graphicx}
\usepackage{centernot}

%--------------------------------------------------------------------
%
%  Comment to avoid compile error
%
%%%%\usepackage{fontawesome}

\usepackage{environ}






\iftoggle{biPrintOnly}{  \input{tcbset_gray}
   }
   {                     %-----------------------------------------------------------------------------------------------------------------------------------------
%
% Function:
%          The setting for printing
%              1. Set the alternate row color to white
%
%
%     
%
% Date: 2020.08.06
%
%-----------------------------------------------------------------------------------------------------------------------------------------

    \colorlet{bkcolor}{gray!15}
	\colorlet{effbkcolor}{white}

    \tcbset{
        styexample/.style={
                           colback=bkcolor,
		                   colframe=blue,
                           boxrule=1pt,
                           colbacktitle=gray!10!white,
                           colback=bkcolor,
                           coltitle=black,
                           colbacklower=white,
                           % colframe=blue,
                           colframe=white,
                           segmentation style={blue, dotted},
                           borderline={0.2mm}{0mm}{blue},
                           fonttitle=\bfseries,
                           sharp corners},
        styskill/.style={
                           colback=bkcolor,
		                   colframe=blue,
                           boxrule=1pt,
                           colbacktitle=gray!10!white,
                           colback=bkcolor,
                           coltitle=black,
                           colbacklower=white,
                           % colframe=blue,
                           colframe=white,
                           segmentation style={blue, dotted},
                           borderline={0.2mm}{0mm}{blue},
                           fonttitle=\bfseries},
        styhighlight/.style={
                           colback=bkcolor,
		                   colframe=blue,
                           boxrule=1pt,
                           colbacktitle=gray!10!white,
                           colback=bkcolor,
                           coltitle=black,
                           colbacklower=white,
                           % colframe=blue,
                           colframe=white,
                           segmentation style={blue, dotted},
                           borderline={0.2mm}{0mm}{blue,dotted},
                           fonttitle=\bfseries,
                           sharp corners},
        styintro/.style={  enhanced,
                           colback=bkcolor, 
		                   frame empty,
						   interior empty,
						   colframe=ForestGreen!50!white,
                           coltitle=ForestGreen!50!black,
						   fonttitle=\bfseries,
						   colbacktitle=ForestGreen!15!white,
                           borderline={0.5mm}{0mm}{ForestGreen!15!white},
                           borderline={0.5mm}{0mm}{ForestGreen!50!white,dashed},
                           attach boxed title to top center={yshift=-3mm},
                           boxed title style={boxrule=0.4pt},
						   varwidth boxed title
						   },
        introductionsty/.style={
                           enhanced,
                           breakable,
                           colback=structurecolor!10,
                           colframe=structurecolor,
                           fonttitle=\bfseries,
                           colbacktitle=structurecolor,
                           fontupper=\itshape,
                           attach boxed title to top center={yshift=-3mm,yshifttext=-1mm},
                           boxrule=0pt,
                           toprule=0.5pt,
                           bottomrule=0.5pt,
                           top=8pt,
                           before skip=8pt,
                           sharp corners
                         },
    }
  
}


%--------------------------------------------------------------------------------------------------------------------------------------------------
% Description:
%     Display common math mistakes
%
% URL:
%     Create custom note boxes
%
%     https://tex.stackexchange.com/questions/482643/create-custom-note-boxes
%
% Date: 2020.05.28
%
%-------------------------------------------------------------------------------------------------------------------------------------------------

\definecolor{redSeven}{rgb}{0.94, 0.24, 0.24}
\definecolor{redNine}{rgb}{0.79, 0.16, 0.16}

% ORANGE
\definecolor{orangeSeven}{rgb}{0.97, 0.4, 0.03}
\definecolor{orangeNine}{rgb}{0.85, 0.28, 0.06}

%    BANS ----------------------------------------------------------------------

\newcommand{\banBox}[2][ban]{%
    \ \bigskip\newline         % <----------------- added \ and space
    \begin{minipage}{.1\textwidth}\centering
        \textcolor{redSeven}{\fontsize{16}{24}\faicon{#1}}
    \end{minipage}%
    \textcolor{redSeven}{\vrule width 3pt \hspace{14pt}}
    \begin{minipage}{.8\textwidth}
        \textcolor{redNine}{#2}
    \end{minipage}
    \smallskip\newline
}

\newenvironment{banEnv}[1][ban]%
{\bigskip\newline%
    \begin{minipage}{.1\textwidth}\centering%
        \textcolor{redSeven}{\fontsize{16}{24}\faicon{#1}}%
    \end{minipage}%
    \textcolor{redSeven}{\vrule width 3pt \hspace{14pt}}%
    \begin{minipage}{.8\textwidth}
        \color{redNine}}
    {\end{minipage}\bigskip\newline}

%    WARNINGS ------------------------------------------------------------------

\newcommand{\warningBox}[2][exclamation-triangle]{%
    \bigskip \newline
    \begin{minipage}{.1\textwidth}\centering
        \textcolor{orangeSeven}{\fontsize{16}{24}\faicon{#1}}
    \end{minipage}%
    \textcolor{orangeSeven}{\vrule width 3pt \hspace{14pt}}
    \begin{minipage}{.8\textwidth}
        \textcolor{orangeNine}{#2}
    \end{minipage}
    \newline
}



\newenvironment{warningEnvPend}[1][exclamation-triangle]%
{
    % \bigskip\newline
    \begin{minipage}{.1\textwidth}\centering%
        \textcolor{orangeSeven}{\fontsize{16}{24}\faicon{#1}}%
    \end{minipage}%
    \textcolor{orangeSeven}{\vrule width 3pt \hspace{14pt}}%
    \begin{minipage}{.8\textwidth}
        %\color{orangeNine}
        }
    {\end{minipage}\bigskip\newline}


\newenvironment{warningEnv}%
{
    % \bigskip\newline
    \begin{minipage}{.1\textwidth}\centering%
        \textcolor{orangeSeven}{\fontsize{16}{24}\faExclamationTriangle}%
    \end{minipage}%
    \textcolor{orangeSeven}{\vrule width 3pt \hspace{14pt}}%
    \begin{minipage}{.8\textwidth}
        %\color{orangeNine}
        }
    {\end{minipage}\bigskip\newline}

%-------------------------------------------------------------------------------



% -------------------------------------------------------------------------------------------------
% Theorem
%       
% https://texblog.org/2015/09/30/fancy-boxes-for-theorem-lemma-and-proof-with-mdframed/
%
% -------------------------------------------------------------------------------------------------

\newcounter{theo}[section] \setcounter{theo}{0}
\renewcommand{\thetheo}{\arabic{section}.\arabic{theo}}
\newenvironment{theo}[2][]{%
\refstepcounter{theo}%
\ifstrempty{#1}%
{\mdfsetup{%
frametitle={%
\tikz[baseline=(current bounding box.east),outer sep=0pt]
\node[anchor=east,rectangle,fill=blue!20]
{\strut Theorem~\thetheo};}}
}%
{\mdfsetup{%
frametitle={%
\tikz[baseline=(current bounding box.east),outer sep=0pt]
\node[anchor=east,rectangle,fill=blue!20]
{\strut Theorem~\thetheo:~#1};}}%
}%
\mdfsetup{innertopmargin=10pt,linecolor=blue!20,%
linewidth=2pt,topline=true,%
frametitleaboveskip=\dimexpr-\ht\strutbox\relax
}
\begin{mdframed}[]\relax%
\label{#2}}{\end{mdframed}}



%-------------------------------------------------------------------------------------------------
% Lemma
%        
% https://texblog.org/2015/09/30/fancy-boxes-for-theorem-lemma-and-proof-with-mdframed/
%
%-------------------------------------------------------------------------------------------------

\newcounter{lem}[section] \setcounter{lem}{0}
\renewcommand{\thelem}{\arabic{section}.\arabic{lem}}
\newenvironment{lem}[2][]{%
\refstepcounter{lem}%
\ifstrempty{#1}%
{\mdfsetup{%
    frametitle={%
    \tikz[baseline=(current bounding box.east),outer sep=0pt]
    \node[anchor=east,rectangle,fill=green!20]
    {\strut Lemma~\thelem};}}
}%
{\mdfsetup{%
    frametitle={%
    \tikz[baseline=(current bounding box.east),outer sep=0pt]
    \node[anchor=east,rectangle,fill=green!20]
    {\strut Lemma~\thelem:~#1};}}%
}%
\mdfsetup{innertopmargin=10pt,linecolor=green!20,%
    linewidth=2pt,topline=true,%
    frametitleaboveskip=\dimexpr-\ht\strutbox\relax
}
\begin{mdframed}[]\relax%
    \label{#2}}{\end{mdframed}}



%-------------------------------------------------------------------------------------------------
% Proof
%        
% https://texblog.org/2015/09/30/fancy-boxes-for-theorem-lemma-and-proof-with-mdframed/
%
%-------------------------------------------------------------------------------------------------
\newcounter{prf}[section]\setcounter{prf}{0}
\renewcommand{\theprf}{\arabic{section}.\arabic{prf}}
\newenvironment{prf}[2][]{%
\refstepcounter{prf}%
\ifstrempty{#1}%
{\mdfsetup{%
frametitle={%
\tikz[baseline=(current bounding box.east),outer sep=0pt]
\node[anchor=east,rectangle,fill=red!20]
{\strut Proof~\theprf};}}
}%
{\mdfsetup{%
frametitle={%
\tikz[baseline=(current bounding box.east),outer sep=0pt]
\node[anchor=east,rectangle,fill=red!20]
{\strut Proof~\theprf:~#1};}}%
}%
\mdfsetup{innertopmargin=10pt,linecolor=red!20,%
linewidth=2pt,topline=true,%
frametitleaboveskip=\dimexpr-\ht\strutbox\relax
}
\begin{mdframed}[]\relax%
\label{#2}}{\qed\end{mdframed}}
%%%%%%%%%%%%%%%%%%%%%%%%%%%%%%







%-------------------------------------------------------------------------------------------------
% Skill
%        
% https://texblog.org/2015/09/30/fancy-boxes-for-theorem-lemma-and-proof-with-mdframed/
%
%-------------------------------------------------------------------------------------------------

\newcounter{skillList}[section] \setcounter{skillList}{0}
\renewcommand{\theskillList}{\arabic{skillList}}
\newenvironment{skillList}[2][]{%
\refstepcounter{skillList}%
\ifstrempty{#1}%
{\mdfsetup{%
frametitle={%
\tikz[baseline=(current bounding box.east),outer sep=0pt]
\node[anchor=east,rectangle,fill=blue!20]
{\strut Skill~\theskillList};}}
}%
{\mdfsetup{%
frametitle={%
\tikz[baseline=(current bounding box.east),outer sep=0pt]
\node[anchor=east,rectangle,fill=blue!20]
{\strut Skill~\theskillList:~#1};}}%
}%
\mdfsetup{innertopmargin=10pt,linecolor=blue!20,%
linewidth=2pt,topline=true,%
frametitleaboveskip=\dimexpr-\ht\strutbox\relax
}
\begin{mdframed}[]\relax%
\label{#2}}{\end{mdframed}}



%-------------------------------------------------------------------------------------------------





%-------------------------------------------------------------------------------------------------------------
%    https://tex.stackexchange.com/questions/28087/example-of-fancy-table-using-tikz-package
%
%
%
%
%-------------------------------------------------------------------------------------------------------------

\tikzset{ 
    table/.style={
        matrix of nodes,
        row sep=-\pgflinewidth,
        column sep=-\pgflinewidth,
        nodes={
            rectangle,
            draw=black,
            align=center
        },
        minimum height=1.5em,
        text depth=0.5ex,
        text height=2ex,
        nodes in empty cells,
        every odd row/.style={
            nodes={fill=gray!20}
        },
        column 1/.style={
            nodes={text width=2em,font=\bfseries}
        },
        row 1/.style={
            nodes={
                fill=gray!40,
                text=black,
                font=\bfseries
            }
        }
    },
}

\tcbset{	
    styleTitle/.style={
	                   enhanced,
                       attach boxed title to top left={yshift=-\tcboxedtitleheight/2,xshift=4ex},
                       colframe=black,
                       colback=white,
                       fonttitle=\bfseries,
                       colbacktitle=white,
                       coltitle=black,
                       boxed title style={
                                           boxrule=0pt,
                                           colframe=white,
                                         },
	},
    styleTip/.style={
	                 enhanced,
	                 before skip=2mm,
    	             after skip=2mm,
                     colback=bkcolor,
                     coltitle=black,
                     colframe=black!50,
                     boxrule=0.2mm,
                     attach boxed title to top left={xshift=1cm,yshift*=1mm-\tcboxedtitleheight},
                     varwidth boxed title*=-3cm,
                     boxed title style={
                           frame code={
                             \path[fill=tcbcolback!30!effbkcolor,
						           draw=blue!50]
                                       ([yshift=-1mm,xshift=-1mm]frame.north west)
                                       arc[start angle=0,end angle=180,radius=1mm]
                                       ([yshift=-1mm,xshift=1mm]frame.north east)
                                       arc[start angle=180,end angle=0,radius=1mm];
                             \path[left color=tcbcolback!60!effbkcolor,
						           right color=tcbcolback!60!effbkcolor,
                                   middle color=tcbcolback!80!effbkcolor,
							       draw=black!50]
                               ([xshift=-2mm]frame.north west) -- ([xshift=2mm]frame.north east)
                               [rounded corners=1mm]-- ([xshift=1mm,yshift=-1mm]frame.north east)
                               -- (frame.south east) -- (frame.south west)
                               -- ([xshift=-1mm,yshift=-1mm]frame.north west)
                               [sharp corners]-- cycle;
                          },
					      interior engine=empty,
                     },
                     fonttitle=\bfseries,
	},
    styleDashed/.style={
        enhanced,
        borderline={0.5pt}{0pt}{dashed,gray},
        before skip=0pt, 
        after skip=0pt,
        boxrule=0.5pt,
        colframe=white, 
        colback=white, 
        arc=0.5ex,
        boxsep=.5ex, 
        top=1.5ex, bottom=0.5ex, left=0.5em, right=0.5em,
        colbacktitle=white, 
        coltitle=black,
        attach boxed title to top left={
                                        xshift=0.7cm, 
                                        yshift =-\tcboxedtitleheight/2
                                        },
        boxed title style={enhanced, 
                           boxrule=0.25pt, 
                           colframe=white
                        },
        fonttitle=\bfseries,
    },
}

\iftoggle{bSolution}{
       \tcbset{styleUnderline/.style={
                                      %-------styleDashed,
		                                enhanced,
                                        borderline={0.5pt}{0pt}{dashed,gray},
                                        before skip=0pt, 
                                        after skip=0pt,
                                        boxrule=0.5pt,
                                        colframe=white, 
                                        colback=white, 
                                        arc=0.5ex,
                                        boxsep=.5ex, 
                                        top=1.5ex, bottom=0.5ex, left=0.5em, right=0.5em,
                                        colbacktitle=white, 
                                        coltitle=black,
                                        attach boxed title to top left={
                                               xshift=0.7cm, 
                                               yshift =-\tcboxedtitleheight/2
                                            },
                                        boxed title style={ enhanced, 
                                                            boxrule=0.25pt, 
                                                            colframe=white
                                                          },
                                        fonttitle=\bfseries,

	                               } 
              } 
    }{
       \tcbset{styleUnderline/.style={ 
                                      enhanced,
									  sharp corners,
									  colframe=white,
									  colback=white,
		                              underlay={\begin{tcbclipinterior} 
		                                                 \draw[help lines,
														       blue,
				                                               densely dotted, 
						                				 	   xstep=0,  
											                   ystep=1.2cm,                                                                 black,
                              shift={(interior.north west)}
															   ]
			                               	                         (interior.north west)+(0,-0.01cm) grid (interior.south east);
		                                        \end{tcbclipinterior}}
	                                 } 
        } 
    }

%----------------------------------------------------------------------------------------
%    PAGE HEADERS
%----------------------------------------------------------------------------------------

\usepackage{fancyhdr} % Required for header and footer configuration

\pagestyle{fancy}
\renewcommand{\chaptermark}[1]{\markboth{\sffamily\normalsize\bfseries\chaptername\ \thechapter.\ #1}{}} % Chapter text font settings
\renewcommand{\sectionmark}[1]{\markright{\sffamily\normalsize\thesection\hspace{5pt}#1}{}} % Section text font settings
\fancyhf{} 
\fancyhead[LE,RO]{\sffamily\normalsize\thepage} % Font setting for the page number in the header
\fancyhead[LO]{\rightmark} % Print the nearest section name on the left side of odd pages
%\fancyhead[CE,CO]{Date: \underline{\hspace{2cm}} Time: \underline{\hspace{1cm}}}
\fancyhead[RE]{\leftmark}  % Print the current chapter name on the right side of even pages
\renewcommand{\headrulewidth}{0.5pt} % Width of the rule under the header
\addtolength{\headheight}{2.5pt} % Increase the spacing around the header slightly
\renewcommand{\footrulewidth}{0pt} % Removes the rule in the footer


%\fancyfoot[LE,RO]{Guo De Liang Thomas}
%\fancyfoot[LO]{\textsuperscript{\textcopyright}2023 Elite Tuition Limited} % Print the nearest section name on the left side of odd pages
%\fancyfoot[RE]{\textsuperscript{\textcopyright}2023 Elite Tuition Limited} % Print the current chapter name on the right side of even pages

% \renewcommand{\footrulewidth}{2.5pt} % Removes the rule in the footer

\renewcommand{\footrulewidth}{0.4pt}

\iftoggle{biPublish}{
       \fancyfoot[L]{\href{http://www.mathxue.com}{\textsuperscript{\textcopyright} mathxue \the\year}}
       \fancyfoot[R]{\href{http://www.mathxue.com}{\textsuperscript{\textcopyright} mathxue \the\year}}
       
       % \fancyfoot[LE,RO]{\href{http://www.eliteducation.net}{\textsuperscript{\textcopyright}Elite Tuition 2021 http://www.eliteducation.net}}
    }
    {
       \fancyfoot[LE,RO]{gipnc63@gmail.com}  
    }


\iftoggle{biTutorCN}{
       \fancyfoot[LE]{\href{http://www.mathxue.com}{\textsuperscript{\textcopyright} 
                                          \begin{CJK*}{UTF8}{gbsn}???? ????\end{CJK*}  }}
       \fancyfoot[RE]{Elite Tuition  \;\;  74 John Gill Road}
        
                      
       \fancyfoot[RO]{\href{http://www.mathxue.com}{\textsuperscript{\textcopyright} Guo DeLiang Thomas \;\; (Mobile: 022 313 4665)     }}
       
      % \fancyfoot[RO]{\href{http://www.mathxue.com}{\textsuperscript{\textcopyright} \begin{CJK*}{UTF8}{gbsn}???\end{CJK*}  Guo DeLiang Thomas Telephone: 022 313 4665     }}
       
       % \fancyfoot[LE,RO]{\href{http://www.mathxue.com}{\textsuperscript{\textcopyright}Elite Tuition 2019 http://www.mathxue.com}}
    }
    {
       \fancyfoot[LE,RO]{gipnc63@gmail.com  }  
    }



\fancypagestyle{plain}{\fancyhead{}\renewcommand{\headrulewidth}{0pt}} % Style for when a plain pagestyle is specified

% Removes the header from odd empty pages at the end of chapters
\makeatletter
\renewcommand{\cleardoublepage}{
\clearpage\ifodd\c@page\else
\hbox{}
\vspace*{\fill}
\thispagestyle{empty}
\newpage
\fi}

                             
\setcounter{secnumdepth}{5}
\setcounter{tocdepth}{4}



%------------------------------------------------------------------------------------
% Description: 
%     To display or hide certain informations based on the following tags:
%
%      tgRefBook    - Referenced book information
%      tgRefWeb     - Referenced web URL and information
%      tgRefExamQ   - Referenced past paper question
%
%      tgExamQCIE   - To display past papers of Cambridge
%      tgExamQNCEA  - To display past papers of NCEA     
%
%      tgShowSolution - To display solution
%
%
%                   - To prepare different version of documents such as textbooks, workbooks
%        
%        
%
%
% URL:
%  
%
% Date(yyyy.mm.dd): 
%     2020.04.04
%
%--------------------------------------------------------------------
\usepackage{tagging}




\usetag{tgRefBook}
\usetag{tgRefExamQ}
\usetag{tgRefWeb} 

\usetag{tgShowSolution}

\newtcbox{\bxtgRefBook}{colframe=black}

%\newtcolorbox{\enRefBook}{colframe=black}



\newenvironment{envRefBook}
{
    \begin{taggedblock}{tgRefBook}
    \begin{tcolorbox}[colframe=blue]
}
{
    \end{tcolorbox}
    \end{taggedblock}
}

%--------------------------------------------------------------------
% Description:
%     Create multiple dotted lines
%
% URL:
%     https://tex.stackexchange.com/questions/248451/how-to-create-multiple-dotted-lines
%
% Date: 2019.07.07
%
%--------------------------------------------------------------------


\newcommand{\Pointilles}[2][3]{%
    \iftoggle{bDottedLine}{
        %\par\nobreak
        %\noindent\rule{0pt}{1.5\baselineskip}% Provides a larger gap between the preceding paragraph and the dots
        %\multido{}{#2}{\noindent\makebox[\linewidth]{\rule{0pt}{#1\baselineskip}\dotfill}\endgraf}% ... dotted lines ...
        \par\nobreak
        \noindent\rule{0pt}{1.5\baselineskip}% Provides a larger gap between the preceding paragraph and the dots
        \multido{}{#2}{\noindent\makebox[\linewidth]{\rule{0pt}{#1\baselineskip}\dotfill}\endgraf}% ... dotted lines ...
        \bigskip% Gap between dots and next paragraph
    }{}  
}





%--------------------------------------------------------------------
% Description:
%     Add page break before each section
%
% Sample:
%     \Pointilles[1.5]{3}
%
% URL:
%     https://tex.stackexchange.com/questions/131015/page-break-after-every-section
%
% Date: 2019.06.26
%
%--------------------------------------------------------------------

\let\oldsection\section
\renewcommand\section{\clearpage\oldsection}



\let\oldsubsection\subsection
\renewcommand\subsection{\clearpage\oldsubsection}

%\let\oldsubsubsection\subsubsection
%\renewcommand\subsubsection{\clearpage\oldsubsubsection}





%-------------------------------------------------------------------------------------------------


%-------------------------------------------------------------------------------------------------



%-------------------------------------------------------------------------------------------------

\providecommand{\cmdIGCSE}{}
\renewcommand{\cmdIGCSE}[9]
{
    %---------------------------------------------------------------------------------------------
    % Exam Questoins by Topic
    % 
    % For example:
    %            0580   - Exam Code  
    %             17     - Year 2017
    %            s      - Summer Paper      
    %            23     - Paper Variance
    %            1      - Question      (For future use)     
    %            5-7    - Pages in Exam
    %            1101   - Difficulty Level         
    %            1      - Code: 
    %                        1. Question
    %                        2. Sample Answer
    %                        3. Mark Scheme
    %            1000   - Skill Code
    %---------------------------------------------------------------------------------------------
    
    \ifnumcomp{1}{>}{1}{#5}{     % -- For future use
    }
    
    \ifnumcomp{#2}{>}{\value{nExamYear}}
    {
        \ifnumcomp{#7}{>}{\value{nLevelStart}}
        {
            \ifnumcomp{#7}{<}{\value{nLevelEnd}}
            {
                \setcounter{nOddEven}{\intcalcMod{#2}{2}}
                \ifnumequal{\value{nWorkBook}}{\value{nOddEven}}{}{ %--- Not Equal
                     \ifnumequal{#8}{1}{
                        \ifboolexpr{togl{biTutor} or togl{biPrintOnly} or togl{biStudent} } { 
                           \include	[scale=0.95
                                       , pages=#6
                                       , pagecommand={\pagestyle{fancy}}
                                       ]
                            {../../Education/ExamPaper/Cambridge_Ages_14_16_IGCSE/Mathematics_0580/0580_#3#2_qp_#4.pdf}
                            
                        }
                     }{}
                }
            }{}
       }{}        
    }{}
}


%---------------------------------------------------------------------------------------------
% Exam Questoins by Topic
%
%      \cmdQuestionByTopic{Page-Range}{Path}
%      
%      
% For example:
%      2-3                        -  Page Range
%      ../../9709_w18_qp_13.pdf   -  Latex File Name include Path (generate by Excel)
%                                    Need to add "" if include space             
%
% Date:
%     2023.04.16
%
%
%---------------------------------------------------------------------------------------------

\newcommand{\cmdQuestionByTopic}[2]
{
    \ifboolexpr{togl{biTutor} or togl{biPrintOnly} or togl{biStudent} } { 
        \IfFileExists{#2}{
            \includepdf[pages=#1
                          , scale = 0.95
                          % , clip=10cm   5cm   5cm  5cm
                          %, trim=0.5cm 0.5cm 0.5cm 11cm
                          %, pagecommand={\pagestyle{fancy}}
                            , frame=true
                        ]{#2}
        }{}                     
    }
}





%---------------------------------------------------------------------------------------------
% Exam Questoins by Topic
%
%      \cmdQuestionCIE{9709}{18}{s}{31}{10i}{5-7}{1110}{1}{Vector, Distance, Point, Line}
%      \cmdQuestionCIE{9709}{w18}  {12}{10}{16-17}{60}{1}
% For example:
%            9709   - Exam Code
%            s17    - Year 2017 Summer Paper             
%            23     - Paper Variance
%            10i    - Question (For future use)     
%            5-7    - Pages in Exam
%            1101   - Difficulty Level         
%            1      - Code: 
%                        1. Question
%                        2. Worked Solution
%                        3. Mark Scheme
%                         
%                         4. Print All
%            1000   - Skill Code
%            Surd   - Skill Description
%
% Date:
%
%---------------------------------------------------------------------------------------------

\providecommand{\cmdQuestionCIE}{}
\renewcommand{\cmdQuestionCIE}[7]
{
0000000000\\ 
    \ifnumequal{#7}{1}{
	    11111111111\\ 
        \ifboolexpr{togl{biTutor} or togl{biPrintOnly} or togl{biStudent} } { 
            \IfFileExists{../../Education/ExamPaper/Cambridge/#1_#2_qp_#3.pdf}{
			     2222222222\\
                \includepdf[pages=#5
                              , scale = 0.95
                              % , clip=10cm   5cm   5cm  5cm
                              %, trim=0.5cm 0.5cm 0.5cm 11cm
                              %, pagecommand={\pagestyle{fancy}}
                                , frame=true
                            ]
                               {../../Education/ExamPaper/Cambridge/#1_#2_qp_#3.pdf}
            }{}                     
        }
    }{}
}





%---------------------------------------------------------------------------------------------
% Get the Cambridge Past Paper. 
%
%
%      \cmdPaperCIE{9709}{m19}{13}
%      \cmdPaperCIE{9709}{w19}{73}
%       
% For example:
%            9709   - Exam Code
%             s17    - Year 2017 Summer Paper             
%            23     - Paper Variance
%            1      - Question      (For future use)     
%            5-7    - Pages in Exam
%            1101   - Difficulty Level         
%            1      - Code: 
%                        1. Question
%                        2. Sample Answer
%                        3. Mark Scheme
%            1000   - Skill Code
%            Surd   - Skill Description
%
% Date:
%
%---------------------------------------------------------------------------------------------

\providecommand{\cmdPaperCIE}{}
\renewcommand{\cmdPaperCIE}[3]
{
    \iftoggle{bQuestion} {
	    %--------------------------------------------------------------------------
		% \ifboolexpr{togl{biTutor} or togl{biPrintOnly} or togl{biStudent} } {
		%
		%--------------------------------------------------------------------------
        \IfFileExists{../../Education/ExamPaper/Cambridge/#1_#2_qp_#3.pdf}{    
            \includepdf[pages=\iPageStart- ,  
                       addtotoc={\iPageStart, subsection, 1, #1\_#2\_qp\_#3, lable\_#1\_#2\_qp\_#3}]
                 {../../Education/ExamPaper/Cambridge/#1_#2_qp_#3.pdf}
        }{}

        \iftoggle{bSolution}{ 
            \IfFileExists{../../Education/ExamPaper/Cambridge/#1_#2_qp_#3_Solution.pdf}{
                \includepdf[pages=1- , frame, scale=0.9, pagecommand={},
                        addtotoc={6, subsubsection, 1, Worked Solution, lable\_#1\_#2\_qp\_#3}]
                    {../../Education/ExamPaper/Cambridge/#1_#2_qp_#3_Solution.pdf}
            }{}
        }

        \iftoggle{bAnswer} { 
            \IfFileExists{../../Education/ExamPaper/Cambridge/#1_#2_ms_#3.pdf}{
                \includepdf[pages=1- 
				            %----- , frame, scale=0.9
							, pagecommand={},
                        addtotoc={6,                %--- Page Number 
						          subsubsection,    %--- Section 
 								  1,                %--- Level
								  Mark Scheme, lable\_#1\_#2\_qp\_#3}]
                    {../../Education/ExamPaper/Cambridge/#1_#2_ms_#3.pdf}
            }{}
        }
    }

	
    \ifboolexpr{(not togl{bQuestion}) and togl {bSolution}}{ 
        \IfFileExists{../../Education/ExamPaper/Cambridge/#1_#2_qp_#3_Solution.pdf}{
            \includepdf[pages=1- , frame, scale=0.9, pagecommand={},
                        addtotoc={1, subsection, 1, #1\_#2\_qp\_#3, lable\_#1\_#2\_qp\_#3,
						          1, subsubsection, 1, Worked Solution, lable\_#1\_#2\_qp\_#3}]
                {../../Education/ExamPaper/Cambridge/#1_#2_qp_#3_Solution.pdf}
        }{}
    }

    \ifboolexpr{(not togl{bQuestion}) and togl {bAnswer}} { 
        \IfFileExists{../../Education/ExamPaper/Cambridge/#1_#2_ms_#3.pdf}{
            \includepdf[pages=1- 
			            %----  , frame, scale=0.9
						, pagecommand={},
                        addtotoc={1, subsection, 1, #1\_#2\_qp\_#3, lable\_#1\_#2\_qp\_#3,
						          6, subsubsection, 1, Mark Scheme, lable\_#1\_#2\_qp\_#3}]
                {../../Education/ExamPaper/Cambridge/#1_#2_ms_#3.pdf}
        }{}
    }

}


%---------------------------------------------------------------------------------------------
% Get the IB Past Paper. 
%
%
%      \cmdPaperIB{IB-Math-HL-Paper1}{2017}{May}{TZ0}{}{} 
%     
%       
% For example:
%            MATHL  - Exam Code
%            HP1    - Year 2017 Summer Paper
%            Year   -
%            Season -             
%            TZ     - Time Zone
%            1      - Question      (For future use)                
%            5-7    - Pages in Exam
%
%
%      Below pending to implement:   
%            1101   - Difficulty Level         
%            1      - Code: 
%                        1. Question
%                        2. Sample Answer
%                        3. Mark Scheme
%                        0. Get Question, Solution, and Mark Scheme
%
%            1000   - Skill Code
%            Surd   -      Description
%
% Date: 
%     2020.04.14
%     
%     
%---------------------------------------------------------------------------------------------


\providecommand{\cmdPaperIB}{}
\renewcommand{\cmdPaperIB}[6]
{

    % \IfSubStr{#6}{-}{  
        % % Includes a - in the paper type
        
        % \StrBefore{#6}{-}[\StrPaper]
        % \StrBehind{#6}{-}[\StrOption]
        
        % The paper is \\
          % \StrPaper  \\
        
        % The option is \\
              % \StrOption        
    % }{}
    
    \StrSubstitute{#1}{-}{_}[\dStrPaper] 

    \iftoggle{bQuestion} {  	
		\IfFileExists{../../Education/ExamPaper/IB/\dStrPaper_#2_#3_#4.pdf}{
				\includepdf[pages=\iPageStart- ,  
						   addtotoc={\iPageStart, subsection, 1, #3 #2  #4  , lable}]
						 {../../Education/ExamPaper/IB/\dStrPaper_#2_#3_#4.pdf}
		}{}


		\iftoggle{bSolution} {     
			\IfFileExists{../../Education/ExamPaper/IB/\dStrPaper_#2_#3_#4_Solution.pdf}{
				\includepdf[pages=1- , 
							% frame, 
							% scale=0.9, 
							pagecommand={},
							addtotoc={1, subsubsection, 1, Worked Solution, lable}]
						 {../../Education/ExamPaper/IB/\dStrPaper_#2_#3_#4_Solution.pdf}
			}{}
		}    

		\iftoggle{bAnswer} { 
			\IfFileExists{../../Education/ExamPaper/IB/\dStrPaper_#2_#3_#4_ms.pdf}{
				\includepdf[pages=1- , 
							% frame, 
							% scale=0.9, 
							pagecommand={},
							addtotoc={1, subsubsection, 1, Mark Scheme, lable}]
						 {../../Education/ExamPaper/IB/\dStrPaper_#2_#3_#4_ms.pdf}
			}{}    
		}
	}

    \ifboolexpr{(not togl{bQuestion}) and togl {bAnswer}} { 
		\IfFileExists{../../Education/ExamPaper/IB/\dStrPaper_#2_#3_#4_ms.pdf}{
			\includepdf[pages=1- , 
							% frame, 
							% scale=0.9, 
							pagecommand={},
							addtotoc={1, subsection,    1, #3 #2  #4  , #3#2#4,
							          1, subsubsection, 1, Mark Scheme, #3#2#4ms}]
						 {../../Education/ExamPaper/IB/\dStrPaper_#2_#3_#4_ms.pdf}
		}{}
		
	}

}



%---------------------------------------------------------------------------------------------
% Get the NCEA Past Paper. 
%
%
%      \cmdPaperNCEA{91261}{2019}
%      \cmdPaperNCEA{91262}{2019}
%       
% For example:
%            9709   - Exam Code
%             2019   - Year 2019             
%      
% Date:
%
%---------------------------------------------------------------------------------------------

\providecommand{\cmdPaperNCEA}{}
\renewcommand{\cmdPaperNCEA}[3][exm]
{
    \ifboolexpr{togl{biTutor} or togl{biPrintOnly} or togl{biStudent} } {
        \IfFileExists{../../Education/ExamPaper/NCEA/#2-#1-#3.pdf}{    
            \includepdf[pages=\iPageStart- ,  
                       addtotoc={\iPageStart, section, 1, Year #2, lable}]
                 {../../Education/ExamPaper/NCEA/#2-#1-#3.pdf}
                 
                 
                 
        }{}
    }

    \ifboolexpr{togl{biTutor} or togl{biStudent} } {     
        \IfFileExists{../../Education/ExamPaper/NCEA/#2-#1-#3-Solution.pdf}{
            \includepdf[pages=1- , 
                        % frame, 
                        % scale=0.9, 
                        pagecommand={},
                        addtotoc={1, subsection, 1, Worked Solution, lable}]
                     {../../Education/ExamPaper/NCEA/#2-#1-#3-Solution.pdf}

        }{}
    }    

    \ifboolexpr{togl{biTutor} or togl{biStudent} } { 
        \IfFileExists{../../Education/ExamPaper/NCEA/#2-ass-#3.pdf}{
            \includepdf[pages=1- , 
                        % frame, 
                        % scale=0.9, 
                        pagecommand={},
                        addtotoc={1, subsection, 1, Mark Scheme, lable}]
                     {../../Education/ExamPaper/NCEA/#2-ass-#3.pdf}
        }{}    
    }

}


%---------------------------------------------------------------------------------------------
% Get the Past Paper by Year. 
%
%
%      \cmdPaperByYear{9709}{m19}{13}
%      \cmdPaperCIE{9709}{w19}{73}
%       
% For example:
%            9709   - Exam Code
%             s17    - Year 2017 Summer Paper             
%            23     - Paper Variance
%            1      - Question      (For future use)     
%            5-7    - Pages in Exam
%            1101   - Difficulty Level         
%            1      - Code: 
%                        1. Question
%                        2. Sample Answer
%                        3. Mark Scheme
%            1000   - Skill Code
%            Surd   - Skill Description
%
% Date:
%
%---------------------------------------------------------------------------------------------

\providecommand{\cmdPaperByYear}{}
\renewcommand{\cmdPaperByYear}[4]
{    
    \iftoggle{bQuestion} {
	    %--------------------------------------------------------------------------
		% \ifboolexpr{togl{biTutor} or togl{biPrintOnly} or togl{biStudent} } {
		%
		%--------------------------------------------------------------------------
        \IfFileExists{../../Education/ExamPaper/#1/#2-#3.pdf}{    
            \includepdf[pages=1- ,  
                       addtotoc={\iPageStart, subsection, 1, #4, lable}]
                 {../../Education/ExamPaper/#1/#2-#3.pdf}
        }{}

        \iftoggle{bSolution}{ 
            \IfFileExists{../../Education/ExamPaper/#1/#2-#3-Solution.pdf}{
                \includepdf[pages=1- , frame, scale=0.9, pagecommand={},
                        addtotoc={1, subsubsection, 1, Worked Solution, lable}]
                    {../../Education/ExamPaper/#1/#2-#3-Solution.pdf}
            }{}
        }

        \iftoggle{bAnswer} { 
            \IfFileExists{../../Education/ExamPaper/#1/#2-#3-ms.pdf}{
                \includepdf[pages=1- , frame, scale=0.9, pagecommand={},
                        addtotoc={1, subsubsection, 1, Mark Scheme, lable}]
                    {../../Education/ExamPaper/#1/#2-#3-ms.pdf}
            }{}
        }
		
        \iftoggle{bAnswer} { 
            \IfFileExists{../../Education/ExamPaper/#1/#2-#3-rpt.pdf}{
                \includepdf[pages=1- , frame, scale=0.9, pagecommand={},
                        addtotoc={1, subsubsection, 1, Examiner Report , lable}]
                    {../../Education/ExamPaper/#1/#2-#3-rpt.pdf}
            }{}
        }
    }

	
    \ifboolexpr{(not togl{bQuestion}) and togl {bSolution}}{ 
        \IfFileExists{../../Education/ExamPaper/#1/#2-#3-Solution.pdf}{
            \includepdf[pages=1- , frame, scale=0.9, pagecommand={},
                        addtotoc={1, subsection, 1, #4, lable,
						          1, subsubsection, 1, Worked Solution, lable}]
                {../../Education/ExamPaper/#1/#2-#3-Solution.pdf}
        }{}
    }

    \ifboolexpr{(not togl{bQuestion}) and togl {bAnswer}} { 
        \IfFileExists{../../Education/ExamPaper/#1/#2-#3-ms.pdf}{
            \includepdf[pages=1- , frame, scale=0.9, pagecommand={},
                        addtotoc={1, subsection, 1, #4, lable,
						          1, subsubsection, 1, Mark Scheme, lable}]
                {../../Education/ExamPaper/#1/#2-#3-ms.pdf}
        }{}
    }

}


%--------------------------------------------------------------------
%
% URL:
%     https://en.wikibooks.org/wiki/LaTeX/Mathematics#Sums_and_integrals
%
% Date: 2020.04.04
%
%--------------------------------------------------------------------

\newcommand{\dd}{\mathop{}\,\mathrm{d}}



\dashundergapssetup{
    ,gap-number-format = \,\textsuperscript{\normalfont
                         (\thegapnumber)}
    ,gap-font = \itshape
    ,gap-numbers = false
    ,teacher-gap-format = dot
    % ,teacher-mode=true 
    ,gap-widen=true
}



%-----------------------------------------------------------------------------------------------------------------------------------------
%
% URL:
%     https://tex.stackexchange.com/questions/332122/dotted-line-with-appropriate-length-for-answering-a-question
%
%
% Function:
%     1. Add dotted line with appropriate length for answering a question.
%     2. If it is not given the dotted line stretches as long as it can (to the end of line).
%
% Sample:
%     \answerline[3cm]
%     \answerline\newline
%
% Date: 2020.04.29
%
%
%
%
%
% \newcommand\answerline{\@ifnextchar[%]
%     \answerlinetowidth\answerlinetoeol 
%     \bigskip  % Gap between dots and next paragraph
% }
%
%
%
%
%-----------------------------------------------------------------------------------------------------------------------------------------

\makeatletter
\newcommand\answerline{\@ifnextchar[%]
    \answerlinetowidth\answerlinetoeol}
\newcommand\answerlinetowidth[1][0pt]
{
    \hbox to #1{\leaders\hbox to \answerdotsep{\hss.\hss}\hfill}
}
\newcommand\answerlinetoeol{\leaders\hbox to \answerdotsep{\hss.\hss}\hfill\strut}
\newcommand\answerdotsep{6pt}
\makeatother


%-----------------------------------------------------------------------------------------------------------------------------------------
%
% URL:
%     https://tex.stackexchange.com/questions/27258/how-do-i-write-underline-text-but-with-a-dotted-line
%
%
% Function:
%     Add dotted line 
%     
%
% Sample:
%     \udensdot{dotty}
%
% Date: 2020.06.18
%
%-----------------------------------------------------------------------------------------------------------------------------------------

\newcommand{\udensdot}[2][\textwidth]{%
	\iftoggle{bAnswer}{
		\tikz[baseline=(todotted.base)]{
			\node[inner sep=1pt,outer sep=0pt] (todotted) {#2};
			%\draw[densely dotted] (todotted.south west) -- (todotted.south east);
			%\draw[densely dotted] (todotted.south west) -- +(0:4cm);
			\draw[densely dotted] (todotted.south west) -- +(0:#1);
		}%
	}
	{
		\tikz[baseline=(todotted.base)]{
			\node[inner sep=1pt,outer sep=0pt] (todotted) {\color{white}{#2}};
			%\draw[densely dotted] (todotted.south west) -- (todotted.south east);
			%\draw[densely dotted] (todotted.south west) -- +(0:4cm);
			\draw[densely dotted] (todotted.south west) -- +(0:#1);
		}%
	}
}%

%-----------------------------------------------------------------------------------------------------------------------------------------
%     https://tex.stackexchange.com/questions/34312/how-to-create-a-command-with-key-values
%
%     https://tex.stackexchange.com/questions/393622/using-key-values-in-newdocumentcommand
%
%
%
%
%-----------------------------------------------------------------------------------------------------------------------------------------

\pgfkeys{
    /envAnswer/.is family, /envAnswer,
    default/.style = {blankline=3, text = Answer:,length=3cm},
    blankline/.store in=\blankline,
    text/.store in =\answer,
    length/.store in=\length,
}



%-----------------------------------------------------------------------------------------------------------------------------------------
%
% URL:
%     https://tex.stackexchange.com/questions/290686/have-a-command-acting-on-the-environment-body
%
%
% Function:
%     Collecting the environment body is easiest with environ package and its \NewEnviron command and \BODY. 
%     
%
% Sample:
%     
%   
%
% Date: 2020.04.30
%
%-----------------------------------------------------------------------------------------------------------------------------------------

\NewEnviron{envAnswer}[1][]
{
    \pgfkeys{/envAnswer,default, #1}
    
    \ifnum \blankline>0
        \ifboolexpr{not ( togl{bSolution} or togl{bCompact}) } {
            \Pointilles[1.5]{\blankline}
			\vspace{0.3cm}
        }
    
    \fi
    
    \iftoggle{bAnswer}{ 
	    %-- To display answer
        \hfill \answer \udensdot[\length]{
                                  \colorbox{green!30!white}{\BODY}
                                }
    }{  %-- Set to white color to hide answer
        \hfill \answer \udensdot[\length]{
                                  \textcolor{white}{\BODY}
                                }
	}

    %%% \ifboolexpr{not (togl{bAnswer} or togl{bCompact}) }{
    %%%    \hfill \answer \answerline[\length]
    %%% }
	
	\vspace{0.8cm}

}
{
}

\NewEnviron{envFillIn}[1][2cm]
{
    \iftoggle{bAnswer}
    { 
        \udensdot[#1]{\colorbox{green!30!white}{\BODY}}
    }
    {
        \udensdot[#1]{\phantom{\BODY}}}
    }
}
{
}


%---------------------------------------------------------------------------------------
% Obsolute
%
%
%
%
%
%---------------------------------------------------------------------------------------

%\NewEnviron{envSolution}
%{
%    \iftoggle{bSolution}
%    {
%         \begin{bxSolution} \BODY
%         \end{bxSolution}
%    }{}
%}



%------------------------------------------------------------------------------
% https://tex.stackexchange.com/questions/24240/is-there-a-simple-way-to-retroactively-add-the-long-prefix-to-a-macros-definit
%
%
%
%
%------------------------------------------------------------------------------
\newtoks\patchtoks    % helper token register
\def\longpatch#1%     % worker macro
  {\let\myoldmac#1%
   \long\def#1##1{\patchtoks={##1}\myoldmac{\the\patchtoks}}}
   
\longpatch{\phantom}


% \NewTColorBox[auto counter,number within=section]{bxSpace}{+!O{}}{%
    % enhanced,
    % sharp corners,
    % %colframe=white,
    % colback=white,
    % coltitle=green!40!black,
    % fonttitle=\bfseries,
    % styleUnderline,
    % label={exercise@\thetcbcounter},
    % attach title to upper=\quad,
    % lowerbox=ignored,
    % savelowerto=solutions/exercise-\thetcbcounter.tex,
    % record={\string\solution{\thetcbcounter}{solutions/exercise-\thetcbcounter.tex}},
    % #1
% }

\NewTColorBox[auto counter,number within=section]{bxSpace}{+!O{}}{%
    styleUnderline,
	%before skip=-0.1cm,
	after skip=2cm,
	noparskip,
	height=6cm,
    label={exercise@\thetcbcounter},
    attach title to upper=\quad,
    lowerbox=ignored
    % savelowerto=solutions/exercise-\thetcbcounter.tex,
    % record={\string\solution{\thetcbcounter}{solutions/exercise-\thetcbcounter.tex}},
    #1
}



\NewDocumentEnvironment{envSolution}{ +b }{%
  \iftoggle{bSolution}{\begin{bxSpace}{\color{blue}#1}\end{bxSpace}}%
                      {%
                        \iftoggle{bCompact}{}{\begin{bxSpace}\phantom{\parbox{\textwidth}{#1}}%
                         \end{bxSpace}%
                        }%
                      }%
  }{}




\NewEnviron{envAnswerInBox}
{
    \iftoggle{bAnswer}{ \colorbox{green!30!white}{\BODY}   }{}
}
{}







\newenvironment{envWorkedSolL}[2][3cm]
{
    \iftoggle{bAnswer}
    { 
       \udensdot[#1]{#2}      
    }
    {
       \Pointilles[1.5]{1}
    }
}	
{
}



\newenvironment{envWorkedSolS}[2][3cm]
{
    \iftoggle{bAnswer}
    { 
       \udensdot[#1]{#2}      
    }
    {
       \Pointilles[1.5]{1}
    }
}
{
}

\newenvironment{envSol}[1][]
{
  \iftoggle{bCompact} 
  {} % Compact, do nothing.
  {  % Not compact mode 
	\cmdUline \textcolor{white}{.}  & \iftoggle{bSolution} {#1}{\phantom{#1}} & \textcolor{white}{.}  \\[0.5cm]
  }
}
{}


%------------------------------------------------------------------------------
% Function:
%     Draw rounded corner densley dotted box.
%
% Reference:
%     https://tex.stackexchange.com/questions/285487/dashed-version-of-boxed
%
% Date:
%     2023.09.24
%------------------------------------------------------------------------------

\newtcbox{\boxAnswerC}[1][]{
  nobeforeafter,
  colback=white,
  boxrule=1pt,
  enhanced jigsaw,
  arc=5pt,
  rounded coners,
  box align=bottom,
  borderline horizontal={1pt}{0pt}{densely dotted},
  borderline vertical={1pt}{0pt}{densely dotted},
  width=5cm,
  #1
}

\newtcbox{\boxAnswer}[1][]{
%\newtcolorbox[]{\boxAnswer}[2][]{
    nobeforeafter,
    %enhanced,
	width=13cm,
    %boxrule=2pt,
	%boxsep=0pt,
	%top=2pt,
	%left=2pt,
	%bottom=2pt,
    %colback=white, 
    %colframe=red, 
    %frame hidden,
    box align=base,
    overlay={\draw[densely dotted, rounded corners=1mm] (frame.south west) rectangle (frame.north east);},
    #1
}


\newtcolorbox{boxAns}[1][]{
    width=6cm,
    nobeforeafter,
    enhanced,
    boxrule=2pt,
	%boxsep=0pt,
	% %top=2pt,
	% %left=2pt,
	bottom=2pt,
    colback=white, 
    % %colframe=red, 
    frame hidden,
    box align=base,
	baseline=2mm,
    overlay={\draw[densely dotted, rounded corners=1mm] (frame.south west) rectangle (frame.north east);},
    #1
}

%----------------------------------------------------------------------------------
% Function:
%    
% URL:
%    https://tex.stackexchange.com/questions/248889/package-for-framed-boxes
%
% Date:
%----------------------------------------------------------------------------------
\newtcolorbox[auto counter]{boxSkill}[2][]{
    title={\bfseries #2},
    enhanced, 
    %-- drop shadow={black!50!white},
    coltitle=black,
    colback=white,
    top=0.3in,
    attach boxed title to top left={
        xshift=1.5em,
        yshift=-\tcboxedtitleheight/2
    },
    boxed title style={size=big,colback=white},
    #1
}


%-----------------------------------------------------------------
% Function:
%     Draw an underline under math forumlae.
% Webpage:
%     https://tex.stackexchange.com/questions/78929/underlining-an-equation-in-an-align-block
% Date:
%     2023.02.26
%-----------------------------------------------------------------



%% code by Andrew Stacey 
% http://tex.stackexchange.com/questions/51582/background-coloring-with-overlay-specification-in-algorithm2e-beamer-package#51582



\newcommand{\cmdUline}{%
	\noindent 
	\tikz[remember picture,overlay,baseline=4ex]
	\draw[densely dotted] (0,0) --(\linewidth, 0) ;}



\newcommand{\cmdSolution}[1]{%
   
	\iftoggle{bSolution} {\cmdUline \textcolor{white}{.} 111  & #1  & \textcolor{white}{.}  \\[0.5cm]}
	                     { % ----------------------------------------------
						   % Not solution, then check if in compact mode 
                           % 						   
						    \iftoggle{bCompact}{ 333 4} %
							                   {\cmdUline 2222 \textcolor{white}{.} & \phantom{#1} & \textcolor{white}{.}  \\[0.5cm]}
				         } 
}




\newcommand{\cmdWorking}[1]{%
	\cmdUline \iftoggle{bSolution} {#1}{\phantom{#1}} \\[0.5cm]
}

\newcommand{\cmdAnswer}[1]{%
	\cmdUline \textcolor{white}{1}  & \iftoggle{bAnswer} {#1}{\phantom{#1}} & \textcolor{white}{1}  \\[0.5cm]
}



\newcounter{cntTask}


\setlist[enumerate]{leftmargin=1.2cm, labelsep=0.5cm,labelwidth=1.5cm}
\settasks{
	    column-sep=1cm,
		counter=cntTask,
	    label = \arabic*.,
		label-offset=1.5em,
    }
    

%-----------------------------------------------------------------------------------------------------------------------------------------
%
% Book:
%     Latex Cookbook
%
%
% Function:
%     
%
% Date: 2020.05.05
%
%-----------------------------------------------------------------------------------------------------------------------------------------





\rowcolors{2}{bkcolor}{white}

\newcommand{\tblhead}[1]{%
    \textcolor{black}{\textbf{#1}}
}

\renewcommand{\arraystretch}{1.5}


%-----------------------------------------------------------------------------------------------------------------------------------------
%
% Package:
%     tcolorbox
%
%
% Function:
%     
%
% Date: 2020.05.10
%
%-----------------------------------------------------------------------------------------------------------------------------------------

\newtcolorbox[auto counter,number within=section]{bxExample}[2][]{%
    styexample,
    skin=bicolor,
    title=#1
}

\newtcolorbox[auto counter,number within=section]{bxExampleA}[2][]{%
    styexample,
    skin=bicolor,
	nobeforeafter,
	#1
}



\newtcolorbox[]{bxHighlight}[2][]{%
    styleHighlight,
    skin=bicolor,
    title=#1
}


\newtcolorbox[auto counter,number within=section]{bxIntro}[2][]{%
    styintro,
    skin=bicolor,
    title=#1
}




% \newtcolorbox[]{YetAnotherTheorem}[2][]{%
    % { enhanced,
	  % frame empty,
	  % interior empty,
	  % colframe=ForestGreen!50!white,
      % coltitle=ForestGreen!50!black,
	  % fonttitle=\bfseries,
	  % colbacktitle=ForestGreen!15!white,
      % borderline={0.5mm}{0mm}{ForestGreen!15!white},
      % borderline={0.5mm}{0mm}{ForestGreen!50!white,dashed},
      % attach boxed title to top center={yshift=-2mm},
      % boxed title style={boxrule=0.4pt},
	  % varwidth boxed title,
	  % title=#1	 
 % }
 
 
% \newcounter{myexample}

% \newtcbtheorem[use counter=myexample]{YetTheorem}{Theorem}%
% {enhanced,frame empty,interior empty,colframe=ForestGreen!50!white,
% coltitle=ForestGreen!50!black,fonttitle=\bfseries,colbacktitle=ForestGreen!15!white,
% borderline={0.5mm}{0mm}{ForestGreen!15!white},
% borderline={0.5mm}{0mm}{ForestGreen!50!white,dashed},
% attach boxed title to top center={yshift=-2mm},
% boxed title style={boxrule=0.4pt},varwidth boxed title}{theo}



%-----------------------------------------------------------------------------------------------------------------------------------------
%
% Package:
%     https://tex.stackexchange.com/questions/81897/how-to-box-the-correct-answer-in-exam-class
%
%
% Function:
%     How to box the correct answer in Exam class?
%
% Date: 2020.08.05

%---------------------------------------------------------------------------

% \newcommand{\tikzmark}[1]{\tikz[overlay,remember picture] \node (#1) {};}

% \newcommand{\DrawSmartBox}[1][red]{%
    % \tikz[overlay,remember picture]{
    % \draw[#1]
      % ($(bl)+(-1.75em,1em)$) rectangle
      % ($(br)+(0.2em,-0.4em)$);}
% }

% % \MyCorrectSmartChoice:
% % #1 optional argument: aspect customization
% % #2 mandatory argument: the answer
% \newcommand{\MyCorrectSmartChoice}[2][blue]{ \tikzmark{bl}#2\tikzmark{br}\DrawSmartBox[#1]}


%-----------------------------------------------------------------------------------------------------------------------------------------
%
% Package:
%     https://tex.stackexchange.com/questions/211914/set-line-spacing-in-tcolorbox
%
%
% Function:
%     One and half spacing
%
% Date: 2020.07.31
%
%---------------------------------------------------------------------------


\newtcolorbox[auto counter,number within=section]{bxSolution}[1][]{%
    skin=bicolor,
    boxrule=1pt,
	borderline={0.5mm}{0mm}{black,dashed},
    %colbacktitle=gray!10!white,
    %colback=gray!10!white,
	colback=white,
    coltitle=black,
    colbacklower=white,
    colframe=white,
    fonttitle=\bfseries,
    %title=Example ~\thetcbcounter: #2,#1
    title=#1
}


\newtcolorbox[auto counter]{bxTip}[2][]{enhanced,
    before skip=2mm,
    after skip=2mm,
    colback=bkcolor,
	coltitle=black,
    colframe=black!50,
    boxrule=0.2mm,
    attach boxed title to top left={xshift=1cm,yshift*=1mm-\tcboxedtitleheight},
    varwidth boxed title*=-3cm,
    boxed title style={frame code={
                        \path[fill=tcbcolback!30!effbkcolor,
						      draw=blue!50]
                               ([yshift=-1mm,xshift=-1mm]frame.north west)
                               arc[start angle=0,end angle=180,radius=1mm]
                               ([yshift=-1mm,xshift=1mm]frame.north east)
                               arc[start angle=180,end angle=0,radius=1mm];
                        \path[left color=tcbcolback!60!effbkcolor,
						      right color=tcbcolback!60!effbkcolor,
                              middle color=tcbcolback!80!effbkcolor,
							  draw=black!50]
                               ([xshift=-2mm]frame.north west) -- ([xshift=2mm]frame.north east)
                               [rounded corners=1mm]-- ([xshift=1mm,yshift=-1mm]frame.north east)
                               -- (frame.south east) -- (frame.south west)
                               -- ([xshift=-1mm,yshift=-1mm]frame.north west)
                               [sharp corners]-- cycle;
                      },interior engine=empty,
    },
    fonttitle=\bfseries,
    title=~\thetcbcounter: #2,    %-------- {\protect\section{#2}},
    #1%
    }


\newtcolorbox{bxTipTable}[3][]{enhanced,
    before skip=2mm,
    after skip=2mm,
    colback=bkcolor,
	coltitle=black,
    colframe=black!50,
    boxrule=0.2mm,
	top=40mm,
    attach boxed title to top left={xshift=1cm,yshift*=1mm-\tcboxedtitleheight},
    varwidth boxed title*=-3cm,
    boxed title style={frame code={
                        \path[fill=tcbcolback!30!effbkcolor,
						      draw=black!50]
                               ([yshift=-1mm,xshift=-1mm]frame.north west)
                               arc[start angle=0,end angle=180,radius=1mm]
                               ([yshift=-1mm,xshift=1mm]frame.north east)
                               arc[start angle=180,end angle=0,radius=1mm];
                        \path[left color=tcbcolback!60!effbkcolor,
						      right color=tcbcolback!60!effbkcolor,
                              middle color=tcbcolback!80!effbkcolor,
							  draw=black!50]
                               ([xshift=-2mm]frame.north west) -- ([xshift=2mm]frame.north east)
                               [rounded corners=1mm]-- ([xshift=1mm,yshift=-1mm]frame.north east)
                               -- (frame.south east) -- (frame.south west)
                               -- ([xshift=-1mm,yshift=-1mm]frame.north west)
                               [sharp corners]-- cycle;
                      },interior engine=empty,
    },
    fonttitle=\bfseries,
	tabularx={#3},
    title={#2},
    #1
    }


\newcolumntype{Y}{>{\raggedleft\arraybackslash}X}% see tabularx
\newtcolorbox{skilltable}[2][]{enhanced jigsaw,
                               breakable, 
                               fonttitle=\bfseries\large,
                               fontupper=\normalsize\sffamily,
                               colback=white,               %yellow!10!white,
                               colframe=gray!90!white,
                               colbacktitle=purple!10!white,
                               coltitle=black,
                               center title,
                               tabularx={#1},
                               title=#2}


\newtcolorbox{bxExamHeader}[1][]{%
    enhanced,
	sharp corners,
	width=\linewidth-6pt,
    enlarge top by=3pt,
	enlarge bottom by=3pt,
    enlarge left by=3pt,
	enlarge right by=3pt,
	frame hidden,
	boxrule=0pt,
	top=1mm,
	bottom=1mm,
    colframe=green!30!black, 
	colbacktitle=green!50!yellow,
    coltitle=black, 
	colback=white,
    borderline={0.5pt}{-0.5pt}{black,dashed},
    borderline={1pt}{-3pt}{black},
	#1
}


\newtcolorbox[auto counter]{bxGeneral}[1][]{%
    #1
}

%-----------------------------------------------------------------------------------------------------------------------------------------
%
% Package:
%     https://tex.stackexchange.com/questions/210928/tcolorbox-how-do-i-align-on-the-baseline-of-the-title
%
%
% Function:
%     Answer in box (tcolorbox)
%
% Date: 
%     2023.08.05
%
% Example:
%     \begin{boxA}[styleDashed]{Title: Answer}
%	  \end{boxA}   
%
%
%---------------------------------------------------------------------------

\newtcolorbox{boxA}[2][]{
    title=#2,
    #1}



%---------------------------------------------------------------------------------------------
% Include book, exam paper and so on
%    
%
% Note:
%      Strictly for internal use or draft copy only.
%
%
%
%      \cmdRef{Page, FileName}
%      
% For example:
%            
%
% Date:
%      2022.10.15
%
%---------------------------------------------------------------------------------------------

\providecommand{\cmdRef}{}
\newcommand{\cmdRefObselete}[2]
{
    \iftoggle{bRef} { 
        \IfFileExists{#2}{
            \includepdf[pages=#1,
                              , scale = 0.95
                              % , clip=10cm   5cm   5cm  5cm
                              % , trim=0.5cm 0.5cm 0.5cm 11cm
                              % , pagecommand={\pagestyle{fancy}}
                                , frame=true
                            ]{#2}
        }{}
	}{}
}

\renewcommand{\cmdRef}[2]
{
    \iftoggle{bRef} { 
        \IfFileExists{#2}{
            \includepdf[#1]{#2}
        }{}
	}{}
}




%-----------------------------------------------------------------------------------------------------------------------------------------
%
% https://tex.stackexchange.com/questions/7032/good-way-to-make-textcircled-numbers
% 
%
%
% Function:
%     
%
% Date: 2020.08.27
%
%---------------------------------------------------------------------------

\newcommand*\circled[1]{\tikz[baseline=(char.base)]{
                                                    \node[shape=circle,draw,inner sep=2pt] (char) {#1};
													}
					   }


%-----------------------------------------------------------------------------------------------------------------------------------------
%
% Function:
%        Output current date as yyyy.mm.dd. 
%        For example:
%                    2021.08.21 
% 
% Webpage: 
%        https://tex.stackexchange.com/questions/152392/date-format-yyyy-mm-dd     
%
% Function:
%     
%
% Date: 2021.08.21
%
%---------------------------------------------------------------------------

\def\myCurrentDate{\leavevmode\hbox{\the\year.\twodigits\month.\twodigits\day}}
\def\twodigits#1{\ifnum#1<10 0\fi\the#1}







%-----------------------------------------------------------------------------------------------------------------------------------------
%
% Function:
%        Define pgStyleSkill using package fancyhdr (Fancy Header/Footer) 

% 
% Webpage: 
%
%     
%
% Date: 2023.04.14
%
%---------------------------------------------------------------------------

\fancypagestyle{pgStyleSkill}{%
    \fancyhf{}
    % %\fancyhead[LE,RO]{\sffamily\normalsize\thepage} % Font setting for the page number in the header
    % %\fancyhead[LO]{\rightmark} % Print the nearest section name on the left side of odd pages
    % %\fancyhead[CE,CO]{Date: \underline{\hspace{2cm}} Time: \underline{\hspace{1cm}}}
    % %\fancyhead[RE]{\leftmark}  % Print the current chapter name on the right side of even pages

    \renewcommand{\headrulewidth}{0pt}
    \fancyfoot[LO,RE]{\cmdWebLink}
    \fancyfoot[LE,RO] {\thepage}
}



%-----------------------------------------------------------------------------------------------------------------------------------------
%
% Original code script is from package Elegantbook
% 
%
%
% Function:
%     
%
% Date: 2020.08.27
%
%---------------------------------------------------------------------------

\newenvironment{introduction}[1][Introduction]{
    \begin{tcolorbox}[styleIntro,title={#1}]
        \begin{multicols}{2}
          \begin{itemize}[label=\textcolor{structurecolor}{\upshape\scriptsize\SquareShadowBottomRight}]}{
          \end{itemize}
        \end{multicols}
    \end{tcolorbox}}
  

\providecommand{\cmdWebLink}{}

\renewcommand{\cmdWebLink}{
	\href{http://www.mathxue.com}{\copyright\, $www.mathxue.com$ }   
}

%-------------------------------------------------------------------------------------------------
%   Description:
%       To easily move the entire tikzpicture
%  
%   Webpage:
%       https://tex.stackexchange.com/questions/447909/how-to-easily-move-an-entire-drawing
%
%   Date:
%       2023.04.14
%-------------------------------------------------------------------------------------------------


\tikzset{shift entire picture/.style n args={2}{execute at end picture={
\pgfmathtruncatemacro{\tmpx}{sign(#1)}
\pgfmathtruncatemacro{\tmpy}{sign(#2)}
\ifnum\tmpx=1
  \ifnum\tmpy=1
   \path[use as bounding box] ([xshift=-#1,yshift=-#2]current bounding box.south west) rectangle 
(current bounding box.north east);
  \else
   \path[use as bounding box] ([xshift=-#1]current bounding box.south west) rectangle 
([yshift=-#2]current bounding box.north east);
  \fi
\else  
  \ifnum\tmpy=1
   \path[use as bounding box] ([yshift=-#2]current bounding box.south west) rectangle 
([xshift=-#1]current bounding box.north east);
  \else
   \path[use as bounding box] (current bounding box.south west) rectangle 
([xshift=-#1,yshift=-#2]current bounding box.north east); 
  \fi
\fi}}}

%-------------------------------------------------------------------------------------------------
%   Description:
%       Not working
%  
%   Webpage:
%       https://liam.page/2015/05/07/make-every-equations-in-your-document-upright/
%
%   Date:
%       2023.04.12
%-------------------------------------------------------------------------------------------------

  
\everymath{\displaystyle}

%-------------------------------------------------------------------------------------------------
%  https://en.wikibooks.org/wiki/LaTeX/Advanced_Mathematics
%
%
%-------------------------------------------------------------------------------------------------
\setlength{\mathindent}{1cm}



%-----------------------------------------------------------
% Setup: xsim 
%
%
%-----------------------------------------------------------

\usepackage{xsim}





%-----------------------------------------------------------------------------------------------------------------
% Function:
%     A key for \xsimsetup to be defined (Require Expl3)
%
% Webpage: 
%     https://tex.stackexchange.com/questions/632962/xsim-shortsolution-appear-after-each-exercise
%
%
%-----------------------------------------------------------------------------------------------------------------

 \ExplSyntaxOn
 \bool_new:N \l_xsim_shortsolution_bool

 \keys_define:nn {xsim}
   { shortsolution/print .bool_set:N = \l_xsim_shortsolution_bool }

 % new environment:
 \NewDocumentEnvironment{shortsolution}{+b}
   {
     \tl_set:NV \ExerciseType \g_xsim_exercise_type_tl
     \tl_set:NV \ExerciseID \g_xsim_exercise_id_tl
     \SetExerciseProperty {shortsolution} {#1}
     \bool_if:NT \l_xsim_shortsolution_bool
       { #1}
   }
   {}
 \ExplSyntaxOff










\xsimsetup{
	exercise/template=simple,
	exercise/the-counter=\arabic{exercise},
	solution/print=true,
	shortsolution/print = true
	% exercise/within = section
}

%\DeclareExerciseEnvironmentTemplate{simple}
%{\par\noindent\textbf{\GetExerciseProperty{counter}}. }
%{\par}

%\DeclareExerciseEnvironmentTemplate{simple}
%  {\item[\XSIMmixedcase{\GetExerciseName}~\GetExerciseProperty{counter}]}
%  {}

\DeclareExerciseEnvironmentTemplate{simple}
    {\item[\GetExerciseProperty{counter}.]}
    {}


\DeclareExerciseEnvironmentTemplate{simple}
    {}
    {}


% new property:
\DeclareExerciseProperty{shortsolution}

% new environment:
\NewDocumentEnvironment{shortsolution}{+b}
{%
	\edef\ExerciseType{\csname g_xsim_exercise_type_tl\endcsname}%
	\edef\ExerciseID{\csname g_xsim_exercise_id_tl\endcsname}%
	\SetExerciseProperty{shortsolution}{#1}%
}
{}

% we'll use a description list for the list of short solutions:
\newcommand\printshortsolutions{%
	\begin{boxA}[styleDashed]{Answer}
		\begin{description}
			\ForEachUsedExerciseByType{%
				\ifnum\ExercisePropertyGet{##1}{##2}{chapter-value}=\value{chapter}
					\ifnum\ExercisePropertyGet{##1}{##2}{section-value}=\value{section}
						\GetExercisePropertyT{shortsolution}
							{\item[##3.]####1}%
					\fi
				\fi				
			}%
		\end{description}
	\end{boxA}
}



%-----------------------------------------------------------
% Below script block is from the manual of XSIM
%
%
%-----------------------------------------------------------

% \DeclareExerciseEnvironmentTemplate{easyitem}
% {\item[\GetExerciseProperty{counter}]}
% {}


% \DeclareExerciseEnvironmentTemplate{boxed}
% {%
	% \tcolorbox[
	% enhanced ,
	% % attach boxed title to top center = {yshift=-7.5pt} ,
	% colback = white , 
	% colbacktitle = white ,
	% coltitle = black , 
	% % colframe = black ,
	% boxed title style = { colframe = black } ,
	% fonttitle=\bfseries,
	% sharp corners=all,
	% breakable,
	% title=
	% \XSIMmixedcase{\GetExerciseName} % -- Comment the exercise name
	% \textbf{\GetExerciseName ~\GetExerciseProperty{counter}}
	% \IfInsideSolutionF{%
		% ~\GetExerciseProperty{counter}%
		% \IfExercisePropertySetT{subtitle}{. \GetExerciseProperty{subtitle}}%
	% }%
	% ]
	
	% \IfInsideSolutionT{%
		% % \tcbsubtitle{Exercise}
		% %     \GetExerciseBody{exercise}  %% Include exercise body
		% \tcbsubtitle{Solution}
	% }
	
% }
% {
	% \Pointilles[1.5]{4}
	% \endtcolorbox
% }



% declare a user command for short answers:
\NewDocumentCommand\answer{m}{%
	\IfSolutionPrintT{%
		\UseExerciseTemplate{begin}{solution}%
		#1%
		\UseExerciseTemplate{end}{solution}%
	}{}%
}




%-----------------------------------------------------------------------------------------------------------------------------------------
%
% Get the current chapter name, section name, subsection name
% 
% Function:
%     
%  https://tex.stackexchange.com/questions/62241/how-to-get-the-current-chapter-name-section-name-subsection-name-etc
%
% Date: 2020.08.27
%
%---------------------------------------------------------------------------

\let\Chaptermark\chaptermark
\def\chaptermark#1{\def\Chaptername{#1}\Chaptermark{#1}}
\let\Sectionmark\sectionmark
\def\sectionmark#1{\def\Sectionname{#1}\Sectionmark{#1}}
\let\Subsectionmark\subsectionmark
\def\subsectionmark#1{\def\Subsectionname{#1}\Subsectionmark{#1}}
\let\Subsubsectionmark\subsubsectionmark
\def\subsubsectionmark#1{\def\Subsubsectionname{#1}\Subsubsectionmark{#1}}


